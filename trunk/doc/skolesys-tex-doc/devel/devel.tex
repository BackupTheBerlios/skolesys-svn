\documentclass{manual}

\title{SkoleSYS Developer Documentation}

\makeindex			% tell \index to actually write the .idx file


\begin{document}

\maketitle

\ifhtml
\chapter*{Front Matter\label{front}}
\fi

\begin{abstract}

\noindent
This document addresses the developer specific issues of SkoleSYS. It is
split up in two main parts - SkoleSYS library documentation and SkoleSYS
distribution tools (disttools).

SkoleSYS Library Documentation coveres the skolesys python package wich
is essentially all SkoleSYS logic e.g. client, server, configuration, SOAP
and more.

SkoleSYS Distribution Tools documents the three tools for deb-packaging,
APT distribution and install CD creation (skolesys-deb, skolesys-apt and 
skolesys-cd)

\end{abstract}

\tableofcontents

\chapter{SkoleSYS Distribution Tools (trunk/disttools)\label{disttools}}
The SkoleSYS Distribution Tools are just convenience tools I have made to make the job of distributing SkoleSYS fast and easy. That way I can keep the main focus on developing the SkoleSYS libraries and GUI. 

The \member{skolesys-apt} script can fetch content via subversion, tarballs or iso-files and place data where you want it redistributed on target filesystems. You can define symlinks and install-scripts (preinst, postinst, prerm, postrm) and setup file permissions. It is also possible to register a scripts to be run by \member{skolesys-apt} after content extraction (ie. pre-bytecompile python scripts). Once the debinfo files (control files for \member{skolesys-apt}) are in place building a file is as easy as executing one command - you can even setup automatic version ticking and group debian packages together by syncronizing their version number.

After building debian packages \member{skolesys-apt} is the tool for building an APT archive. You only need to setup one aptinfo file (control file for \member{skolesys-apt}) per distribution you target create a \member{distribution hieararchi}\footnote{A distribution hieararchi tells \member{skolesys-apt} which packages go into which distribution components and which computer architecture they are build for.}. \member{skolesys-apt} will automatically generate the \member{dists} and \member{pool} directory so they are ready to put on your distribution webserver.
\section{skolesys-deb\label{skolesys-deb}}
This tool is used to create deb-files for distribution. It can fetch distribution content using svn, iso or tgz. Each deb-file is controlled by a debinfo file, the naming convention for such a file is \member{<module_name>_debinfo.py} - e.g. the module named \member{"python-skolesys-client_feisty"} should be named: \member{"python-skolesys-client_feisty_debinfo.py"}. Module name should not be confused with package name, the package name is specified inside the debinfo file.

\index{skolesys-deb}
\member{Usage: skolesys-deb [options] module_name}


\begin{tableiii}{l|l|l}{command}{Short-option}{Long-option}{Description}
  \lineiii{-r \var{url}}{\member{--svnbase=\var{url}}}{URL to the subversion repository to fetch from (use when \member{fetch_method='svn'})}
  \lineiii{\member{-l \var{dir}}}{\member{--iso-location=\var{dir}}}{The directory containing the iso file (use when \member{fetch_method='iso'})}
  \lineiii{\member{-l \var{dir}}}{\member{--iso-location=\var{dir}}}{The directory containing the tar.gz file (use when \member{fetch_method='tgz'})}
  \lineiii{}{\member{--dont-fetch}}{Don't fetch the package contents. This is used if more than one package is based on the same content resource, so if these packages are build successively there is no need to fetch the content more than once.}
  \lineiii{\member{-h}}{\member{--help}}{Show the help message and exit}
\end{tableiii}
\citetitle{skolesys-deb command line options}
\subsection{debinfo files\label{debinfo}}
Describing the naming convention has already revealed that the python interpreter is used for parsing debinfo files, and that is also why I say module name instead of package name. A debinfo file consists of a number of global variables, most of these are optional but there are three mandatory:

\index{debinfo file options}
\begin{tableiii}{c|l|l}{textrm}{Name}{Type}{Description}
  \lineiii{\member{fetch_method}}{\member{string}}{Specify how to fetch the content for the deb-package.}
  \lineiii{\member{control}}{\member{dict}}{This dictionary holds the normal dpkg control structure. There is a special feature regarding the Version value for grouping packages together giving them the same package version. Simply set the value to point to the file containing the version number in the first line. (e.g. 'Version': 'file://ver_no')}
  \lineiii{\member{copy}}{\member{dict}}{Specify which files should go where in the filesystem during installation.}
\end{tableiii}
\citetitle{Mandatory variables}

\begin{tableiii}{c|l|l}{textrm}{Name}{Type}{Description}
  \lineiii{\member{svn_module}}{\member{string}}{When using subversion as fetch method you can use this variable to specify a certain path in the repository. If this variable is not set skolesys-deb will try to use the module_name as path.}
  \lineiii{\member{svn_repos}}{\member{string}}{Specify the subversion repository to fetch the package source from. If this variable is set it will override the SKOLESYS_SVNBASE environment variable but not the command line argument -r.}
  \lineiii{\member{prebuild_script}}{\member{string}}{Script to be executed after content extraction from svn, iso or tgz and before the deb-file creation (see \member{skolesys-qt3_dapper_debinfo.py}) }
  \lineiii{\member{links}}{\member{dict}}{Each key-value entry in the links dict specifies a symbolic link to be created in the filesystem after installation.}
  \lineiii{\member{perm}}{\member{dict}}{The perm variable can be used to setup file permissions for the files being installed.}
  \lineiii{\member{preinst}}{\member{string}}{Script to be executed before package installation.}
  \lineiii{\member{postinst}}{\member{string}}{Script to be executed after package installation.}
  \lineiii{\member{prerm}}{\member{string}}{Script to be executed before package removal.}
  \lineiii{\member{postrm}}{\member{string}}{Script to be executed after package removal.}
\end{tableiii}
\citetitle{Optional Variables}

\index{debinfo example}
Example:
\begin{verbatim}
fetch_method = "svn"
svn_module = "system"

control = {
	'Package': 'python-skolesys-client',
	'Version': 'file://skolesys_ver',
	'NameExtension': 'feisty_all',
	'Section': 'python',
	'Priority': 'optional',
	'Architecture': 'all',
	'Depends': 'python-support (>= 0.2), python-soappy, python-m2crypto',
	'Recommends': 'skolesys-qt4',
	'Maintainer': 'Jakob Simon-Gaarde <jakob@skolesys.dk>',
	'Replaces': 'python-skolesys-seeder, python2.4-skolesys-seeder, python2.4-skolesys-client',
	'Conflicts': 'python2.4-skolesys-mainserver, python2.4-skolesys-seeder, python-skolesys-mainserver, python-skolesys-seeder',
	'Provides': 'python2.5-skolesys-client, python2.3-skolesys-client, python2.4-skolesys-client',
	'Description': 'This is the soap client part of the SkoleSYS linux distribution',
	'longdesc': 
""" The skolesys package provides the nessecary tools for administrating the SkoleSYS
 distribution. The main issue here is creating users and groups, controlling permissions,
 creating user and group spaces, registering client workstations (Windows, Linux, MacOS)
 and registering thin client servers (LTSP).
"""}

perm = {'cfmachine/cfinstaller.py': '755',
	'soap/getconf.py': '755',
	'soap/reghost.py': '755'}

copy = {
	'__init__.py': '/usr/share/python-support/python-skolesys-client/skolesys/',
	'soap/__init__.py': '/usr/share/python-support/python-skolesys-client/skolesys/soap',
	'soap/netinfo.py': '/usr/share/python-support/python-skolesys-client/skolesys/soap',
	'soap/marshall.py': '/usr/share/python-support/python-skolesys-client/skolesys/soap',
	'soap/getconf.py': '/usr/share/python-support/python-skolesys-client/skolesys/soap',
	'soap/reghost.py': '/usr/share/python-support/python-skolesys-client/skolesys/soap',
	'soap/client.py': '/usr/share/python-support/python-skolesys-client/skolesys/soap',
	'soap/p2.py': '/usr/share/python-support/python-skolesys-client/skolesys/soap',
	'cfmachine/__init__.py': '/usr/share/python-support/python-skolesys-client/skolesys/cfmachine',
	'cfmachine/cfinstaller.py': '/usr/share/python-support/python-skolesys-client/skolesys/cfmachine',
	'cfmachine/apthelpers.py': '/usr/share/python-support/python-skolesys-client/skolesys/cfmachine',
	'cfmachine/fstabhelpers.py': '/usr/share/python-support/python-skolesys-client/skolesys/cfmachine',
	'tools': '/usr/share/python-support/python-skolesys-client/skolesys/',
	'definitions': '/usr/share/python-support/python-skolesys-client/skolesys/'}

links = {
	'/usr/sbin/ss_installer': '../share/python-support/python-skolesys-client/skolesys/cfmachine/cfinstaller.py',
	'/usr/sbin/ss_getconf': '../share/python-support/python-skolesys-client/skolesys/soap/getconf.py',
	'/usr/sbin/ss_reghost': '../share/python-support/python-skolesys-client/skolesys/soap/reghost.py'}

postinst = """#!/bin/sh
set -e
# Automatically added by dh_pysupport
if [ "$1" = "configure" ] && which update-python-modules >/dev/null 2>&1; then
        update-python-modules -i /usr/share/python-support/python-skolesys-client
fi
# End automatically added section
"""

prerm = """#!/bin/sh
set -e
# Automatically added by dh_pysupport
if which update-python-modules >/dev/null 2>&1; then
        update-python-modules -c -i /usr/share/python-support/python-skolesys-client
fi
# End automatically added section
"""

postrm = """#!/bin/sh
if [ -e /usr/share/python-support/python-skolesys-client/skolesys ]
then
  find /usr/share/python-support/python-skolesys-client/skolesys -name "*.pyc" -delete
  find /usr/share/python-support/python-skolesys-client/skolesys -name "*.pyo" -delete
fi
"""
\end{verbatim}

\subsection{Subversion based modules\label{skolesys-deb-svn}}
Most of the skolesys-deb modules rely on fetching their content from a subversion repository. It is possible to setup this repository once and for all by setting the SKOLESYS_SVNBASE environment variable. 

However, if you need to fetch content for different skolesys-deb modules from different repositories you can specify the repository for each module by setting the svn_repos variable in the debinfo file or you can do it by using the -r command line option. svn_repos will override SKOLESYS_SVNBASE, -r command line options overrides everything.
The debinfo variable svn_module specifies the path to fetch from inside the repository. If svn_module is unset skolesys-deb will try to fetch using the module_name as path.

\begin{tableii}{c|l}{}{Environment Variable}{Description}
  \lineii{\member{SKOLESYS_SVNBASE}}{The default subversion repository or repository parent dir.}
\end{tableii}

\note{If you need to fetch from the base of a repository you will have to set svn_repos to the parent dir of the repository - so if I need to fetch from the root of the repository located at svn.mydomain.org/srv/svn/skolesys I will need to set svn_repos to somthing like svn+ssh://svn.mydomain.org/srv/svn/ and svn_module to skolesys.}

\subsection{ISO and tar.gz based modules\label{skolesys-deb-iso}}
There is not much to say about these fetch methods. All skolesys-deb needs to know is where to find the iso or tgz files. Like when using subversion for fetch method you can set environment variables that will serve as default directories for iso/tgz file hunting.

command line options -l iso_dir and -t  tgz_dir can also be used to set the directory location and these will of course override the environment variables if set. 

\begin{tableii}{c|l}{}{Environment Variable}{Description}
  \lineii{\member{SKOLESYS_ISODIR}}{The default directory when searching for ISO files.}
  \lineii{\member{SKOLESYS_TGZDIR}}{The default directory when searching for tgz files.}
\end{tableii}

\section{skolesys-apt\label{skolesys-apt}}
After creating debian packages the next thing needed is to make life easier for debian users who want to use the packages. To do that we are going to create a standard debian package repository which debian APT (Advanced Package Tool) relies on. 

skolesys-apt can do this automatically, all we need to do is create a simple aptinfo file and place the debian files in a distribution hierarchi - also we need a gpg key-pair to sign the APT Release file. If you don't have such a key-pair it will be shown how to gererate it.

skolesys-apt will create the APT repository on the local filesystem ready to ship by copying the two folders \member{dists} and \member{pool} and the public part of the gpg key-pair to the host running the webserver that is going to expose the repository.

\index{skolesys-apt}
\member{Usage: skolesys-apt dist_name}

\subsection{Prepair gpg key-pair\label{skolesys-apt-gpg}}

Prerequisite: You need \member{gnupg} installed to go any further.

\subsubsection{Reuse an existing gpg key-pair\label{skolesys-apt-reusegpg}}
If you already have a gpg key-pair you wish to use for signing the APT repository it os possible to export the key-pair and then import it on the host where you are running skolesys-apt.
Let us say that the user gpg-owner on host gpg-source-host has the key-pair we want to export and the user skolesys-devel on host gpg-dest-host needs to import it:

Export key-pair:
\begin{enumerate}
 \item log on to the source host gpg-owner@gpg-source-host
 \item Export the gpg key-pair:
\begin{verbatim}
gpg-owner@gpg-source-host$ gpg --armor --export-secret-keys EEF2B7FA > secret.gpg.asc
gpg-owner@gpg-source-host$ gpg --armor --export EEF2B7FA > public.gpg.asc
\end{verbatim}
\end{enumerate}
\note{EEF2B7FA is the key ID. Use gpg --list-keys to view your key ID's}

Import key-pair:

\begin{enumerate}
 \item Copy the secret.gpg.asc and public.gpg.asc from the source host
 \item log on to the destination host skolesys-developer@gpg-dest-host
 \item Import the key-pair
\begin{verbatim}
skolesys-devel@gpg-dest-host$ gpg --allow-secret-key-import --import public.gpg.asc secret.gpg.asc
\end{verbatim}
\end{enumerate}

\member{public.gpg.asc} will be the public key block that you must expose to the public to make your signed APT repository validate and thereby accessible to remote debian based systems.

\subsubsection{Create a gpg key-pair\label{skolesys-apt-creategpg}}
If you don't have a gpg key-pair to use for this task then you need to create one now.
\begin{enumerate}
 \item Create the gpg key-pair
 \begin{verbatim}
gpg --gen-key

.... (I used the no-existing email: jakob@email.com)

pub   1024D/55117311 2007-06-21
	Key fingerprint = F461 E5AF 0DE8 B3E4 358D  DEB6 EC23 7CF4 5511 7311
uid                  Jakob Simon-Gaarde <jakob@email.com>
sub   2048g/FDD61F6E 2007-06-21
 \end{verbatim}
 \note{It is not required that you supply a passphrase you can leave it empty.}

 \item In the example above the newly generated key-pair has ID 55117311
 \item Extract a public key block	
 \begin{verbatim}
gpg --export --armor jakob@email.com > public.gpg.asc
 \end{verbatim}

 \member{public.gpg.asc} will be the public key block that you must expose to the public to make your signed APT repository validate and thereby accessible to remote debian based systems.

\end{enumerate}

\subsection{Creating a distribution hierarchi\label{skolesys-apt-pool}}
Most of the skolesys-deb modules rely on fetching their content from a subversion repository. It is possible to setup this repository once and for all by setting the SKOLESYS_SVNBASE environment variable. 

However, if you need to fetch content for different skolesys-deb modules from different repositories you can specify the repository for each module by setting the svn_repos variable in the debinfo file or you can do it by using the -r command line option. svn_repos will override SKOLESYS_SVNBASE, -r command line options overrides everything.
The debinfo variable svn_module specifies the path to fetch from inside the repository. If svn_module is unset skolesys-deb will try to fetch using the module_name as path.

\begin{tableii}{c|l}{}{Environment Variable}{Description}
  \lineii{\member{SKOLESYS_SVNBASE}}{The default subversion repository or repository parent dir.}
\end{tableii}

\note{If you need to fetch from the base of a repository you will have to set svn_repos to the parent dir of the repository - so if I need to fetch from the root of the repository located at svn.mydomain.org/srv/svn/skolesys I will need to set svn_repos to somthing like svn+ssh://svn.mydomain.org/srv/svn/ and svn_module to skolesys.}

\subsection{Creating aptinfo file\label{skolesys-apt-aptinfo}}
There is not much to say about these fetch methods. All skolesys-deb needs to know is where to find the iso or tgz files. Like when using subversion for fetch method you can set environment variables that will serve as default directories for iso/tgz file hunting.

command line options -l iso_dir and -t  tgz_dir can also be used to set the directory location and these will of course override the environment variables if set. 

\begin{tableii}{c|l}{}{Environment Variable}{Description}
  \lineii{\member{SKOLESYS_ISODIR}}{The default directory when searching for ISO files.}
  \lineii{\member{SKOLESYS_TGZDIR}}{The default directory when searching for tgz files.}
\end{tableii}

%\section{skolesys-cd\label{skolesys-cd}}
This tool is used to create deb-files for distribution. It can fetch distribution content using svn, iso or tgz. Each deb-file is controlled by a debinfo file, the naming convention for such a file is \member{<module_name>_debinfo.py} - e.g. the module named \member{"python-skolesys-client_feisty"} should be named: \member{"python-skolesys-client_feisty_debinfo.py"}. Module name should not be confused with package name, the package name is specified inside the debinfo file.

\begin{verbatim}
skolesys-deb[-r svn_repos][-l iso_dir][-t tgz_dir][--dont-fetch] module_name
\end{verbatim}


\begin{tableiii}{l|l|l}{command}{Short-option}{Long-option}{Description}
  \lineiii{-r \var{url}}{\member{--svnbase=\var{url}}}{URL to the subversion repository to fetch from (use when \member{fetch_method='svn'})}
  \lineiii{\member{-l \var{dir}}}{\member{--iso-location=\var{dir}}}{The directory containing the iso file (use when \member{fetch_method='iso'})}
  \lineiii{\member{-l \var{dir}}}{\member{--iso-location=\var{dir}}}{The directory containing the tar.gz file (use when \member{fetch_method='tgz'})}
  \lineiii{}{\member{--dont-fetch}}{Don't fetch the package contents. This is used if more than one package is based on the same content resource, so if these packages are build successively there is no need to fetch the content more than once.}
  \lineiii{\member{-h}}{\member{--help}}{Show the help message and exit}
\end{tableiii}

\subsection{debinfo files\label{debinfo}}
Describing the naming convention has already revealed that the python interpreter is used for parsing debinfo files, and that is also why I say module name instead of package name. A debinfo file consists of a number of global variables, most of these are optional but there are three mandatory.

Mandatory variables:
\begin{tableiii}{c|l|l}{textrm}{Name}{Type}{Description}
  \lineiii{\member{fetch_method}}{\member{string}}{Specify how to fetch the content for the deb-package.}
  \lineiii{\member{control}}{\member{dict}}{This dictionary holds the normal dpkg control structure. There is a special feature regarding the Version value for grouping packages together giving them the same package version. Simply set the value to point to the file containing the version number in the first line. (e.g. 'Version': 'file://ver_no')}
  \lineiii{\member{copy}}{\member{dict}}{Specify which files should go where in the filesystem during installation.}
\end{tableiii}

Optional Variables:
\begin{tableiii}{c|l|l}{textrm}{Name}{Type}{Description}
  \lineiii{\member{svn_module}}{\member{string}}{When using subversion as fetch method you can use this variable to specify a certain path in the repository. If this variable is not set skolesys-deb will try to use the module_name as path.}
  \lineiii{\member{svn_repos}}{\member{string}}{Specify the subversion repository to fetch the package source from. If this variable is set it will override the SKOLESYS_SVNBASE environment variable but not the command line argument -r.}
  \lineiii{\member{links}}{\member{dict}}{Each key-value entry in the links dict specifies a symbolic link to be created in the filesystem after installation.}
  \lineiii{\member{perm}}{\member{dict}}{The perm variable can be used to setup file permissions for the files being installed.}
  \lineiii{\member{preinst}}{\member{string}}{Script to be executed before package installation.}
  \lineiii{\member{postinst}}{\member{string}}{Script to be executed after package installation.}
  \lineiii{\member{prerm}}{\member{string}}{Script to be executed before package removal.}
  \lineiii{\member{postrm}}{\member{string}}{Script to be executed after package removal.}
\end{tableiii}

Example:
\begin{verbatim}
fetch_method = "svn"
svn_module = "system"

control = {
	'Package': 'python-skolesys-client',
	'Version': 'file://skolesys_ver',
	'NameExtension': 'feisty_all',
	'Section': 'python',
	'Priority': 'optional',
	'Architecture': 'all',
	'Depends': 'python-support (>= 0.2), python-soappy, python-m2crypto',
	'Recommends': 'skolesys-qt4',
	'Maintainer': 'Jakob Simon-Gaarde <jakob@skolesys.dk>',
	'Replaces': 'python-skolesys-seeder, python2.4-skolesys-seeder, python2.4-skolesys-client',
	'Conflicts': 'python2.4-skolesys-mainserver, python2.4-skolesys-seeder, python-skolesys-mainserver, python-skolesys-seeder',
	'Provides': 'python2.5-skolesys-client, python2.3-skolesys-client, python2.4-skolesys-client',
	'Description': 'This is the soap client part of the SkoleSYS linux distribution',
	'longdesc': 
""" The skolesys package provides the nessecary tools for administrating the SkoleSYS
 distribution. The main issue here is creating users and groups, controlling permissions,
 creating user and group spaces, registering client workstations (Windows, Linux, MacOS)
 and registering thin client servers (LTSP).
"""}

perm = {'cfmachine/cfinstaller.py': '755',
	'soap/getconf.py': '755',
	'soap/reghost.py': '755'}

copy = {
	'__init__.py': '/usr/share/python-support/python-skolesys-client/skolesys/',
	'soap/__init__.py': '/usr/share/python-support/python-skolesys-client/skolesys/soap',
	'soap/netinfo.py': '/usr/share/python-support/python-skolesys-client/skolesys/soap',
	'soap/marshall.py': '/usr/share/python-support/python-skolesys-client/skolesys/soap',
	'soap/getconf.py': '/usr/share/python-support/python-skolesys-client/skolesys/soap',
	'soap/reghost.py': '/usr/share/python-support/python-skolesys-client/skolesys/soap',
	'soap/client.py': '/usr/share/python-support/python-skolesys-client/skolesys/soap',
	'soap/p2.py': '/usr/share/python-support/python-skolesys-client/skolesys/soap',
	'cfmachine/__init__.py': '/usr/share/python-support/python-skolesys-client/skolesys/cfmachine',
	'cfmachine/cfinstaller.py': '/usr/share/python-support/python-skolesys-client/skolesys/cfmachine',
	'cfmachine/apthelpers.py': '/usr/share/python-support/python-skolesys-client/skolesys/cfmachine',
	'cfmachine/fstabhelpers.py': '/usr/share/python-support/python-skolesys-client/skolesys/cfmachine',
	'tools': '/usr/share/python-support/python-skolesys-client/skolesys/',
	'definitions': '/usr/share/python-support/python-skolesys-client/skolesys/'}

links = {
	'/usr/sbin/ss_installer': '../share/python-support/python-skolesys-client/skolesys/cfmachine/cfinstaller.py',
	'/usr/sbin/ss_getconf': '../share/python-support/python-skolesys-client/skolesys/soap/getconf.py',
	'/usr/sbin/ss_reghost': '../share/python-support/python-skolesys-client/skolesys/soap/reghost.py'}

postinst = """#!/bin/sh
set -e
# Automatically added by dh_pysupport
if [ "$1" = "configure" ] && which update-python-modules >/dev/null 2>&1; then
        update-python-modules -i /usr/share/python-support/python-skolesys-client
fi
# End automatically added section
"""

prerm = """#!/bin/sh
set -e
# Automatically added by dh_pysupport
if which update-python-modules >/dev/null 2>&1; then
        update-python-modules -c -i /usr/share/python-support/python-skolesys-client
fi
# End automatically added section
"""

postrm = """#!/bin/sh
if [ -e /usr/share/python-support/python-skolesys-client/skolesys ]
then
  find /usr/share/python-support/python-skolesys-client/skolesys -name "*.pyc" -delete
  find /usr/share/python-support/python-skolesys-client/skolesys -name "*.pyo" -delete
fi
"""
\end{verbatim}

\subsection{Subversion based modules\label{skolesys-deb-svn}}
Most of the skolesys-deb modules rely on fetching their content from a subversion repository. It is possible to setup this repository once and for all by setting the SKOLESYS_SVNBASE environment variable. 

However, if you need to fetch content for different skolesys-deb modules from different repositories you can specify the repository for each module by setting the svn_repos variable in the debinfo file or you can do it by using the -r command line option. svn_repos will override SKOLESYS_SVNBASE, -r command line options overrides everything.
The debinfo variable svn_module specifies the path to fetch from inside the repository. If svn_module is unset skolesys-deb will try to fetch using the module_name as path.

\begin{tableii}{c|l}{}{Environment Variable}{Description}
  \lineii{\member{SKOLESYS_SVNBASE}}{The default subversion repository or repository parent dir.}
\end{tableii}

\note{If you need to fetch from the base of a repository you will have to set svn_repos to the parent dir of the repository - so if I need to fetch from the root of the repository located at svn.mydomain.org/srv/svn/skolesys I will need to set svn_repos to somthing like svn+ssh://svn.mydomain.org/srv/svn/ and svn_module to skolesys.}

\subsection{ISO and tar.gz based modules\label{skolesys-deb-iso}}
There is not much to say about these fetch methods. All skolesys-deb needs to know is where to find the iso or tgz files. Like when using subversion for fetch method you can set environment variables that will serve as default directories for iso/tgz file hunting.

command line options -l iso_dir and -t  tgz_dir can also be used to set the directory location and these will of course override the environment variables if set. 

\begin{tableii}{c|l}{}{Environment Variable}{Description}
  \lineii{\member{SKOLESYS_ISODIR}}{The default directory when searching for ISO files.}
  \lineii{\member{SKOLESYS_TGZDIR}}{The default directory when searching for tgz files.}
\end{tableii}



% \chapter{Debugging \label{debugging}}
%
% XXX Explain Py_DEBUG, Py_TRACE_REFS, Py_REF_DEBUG.

\chapter{SkoleSYS Package (trunk/system)\label{skolesys-pack}}
\section{skolesys-deb\label{skolesys-deb}}
This tool is used to create deb-files for distribution. It can fetch distribution content using svn, iso or tgz. Each deb-file is controlled by a debinfo file, the naming convention for such a file is \member{<module_name>_debinfo.py} - e.g. the module named \member{"python-skolesys-client_feisty"} should be named: \member{"python-skolesys-client_feisty_debinfo.py"}. Module name should not be confused with package name, the package name is specified inside the debinfo file.

\index{skolesys-deb}
\member{Usage: skolesys-deb [options] module_name}


\begin{tableiii}{l|l|l}{command}{Short-option}{Long-option}{Description}
  \lineiii{-r \var{url}}{\member{--svnbase=\var{url}}}{URL to the subversion repository to fetch from (use when \member{fetch_method='svn'})}
  \lineiii{\member{-l \var{dir}}}{\member{--iso-location=\var{dir}}}{The directory containing the iso file (use when \member{fetch_method='iso'})}
  \lineiii{\member{-l \var{dir}}}{\member{--iso-location=\var{dir}}}{The directory containing the tar.gz file (use when \member{fetch_method='tgz'})}
  \lineiii{}{\member{--dont-fetch}}{Don't fetch the package contents. This is used if more than one package is based on the same content resource, so if these packages are build successively there is no need to fetch the content more than once.}
  \lineiii{\member{-h}}{\member{--help}}{Show the help message and exit}
\end{tableiii}
\citetitle{skolesys-deb command line options}
\subsection{debinfo files\label{debinfo}}
Describing the naming convention has already revealed that the python interpreter is used for parsing debinfo files, and that is also why I say module name instead of package name. A debinfo file consists of a number of global variables, most of these are optional but there are three mandatory:

\index{debinfo file options}
\begin{tableiii}{c|l|l}{textrm}{Name}{Type}{Description}
  \lineiii{\member{fetch_method}}{\member{string}}{Specify how to fetch the content for the deb-package.}
  \lineiii{\member{control}}{\member{dict}}{This dictionary holds the normal dpkg control structure. There is a special feature regarding the Version value for grouping packages together giving them the same package version. Simply set the value to point to the file containing the version number in the first line. (e.g. 'Version': 'file://ver_no')}
  \lineiii{\member{copy}}{\member{dict}}{Specify which files should go where in the filesystem during installation.}
\end{tableiii}
\citetitle{Mandatory variables}

\begin{tableiii}{c|l|l}{textrm}{Name}{Type}{Description}
  \lineiii{\member{svn_module}}{\member{string}}{When using subversion as fetch method you can use this variable to specify a certain path in the repository. If this variable is not set skolesys-deb will try to use the module_name as path.}
  \lineiii{\member{svn_repos}}{\member{string}}{Specify the subversion repository to fetch the package source from. If this variable is set it will override the SKOLESYS_SVNBASE environment variable but not the command line argument -r.}
  \lineiii{\member{prebuild_script}}{\member{string}}{Script to be executed after content extraction from svn, iso or tgz and before the deb-file creation (see \member{skolesys-qt3_dapper_debinfo.py}) }
  \lineiii{\member{links}}{\member{dict}}{Each key-value entry in the links dict specifies a symbolic link to be created in the filesystem after installation.}
  \lineiii{\member{perm}}{\member{dict}}{The perm variable can be used to setup file permissions for the files being installed.}
  \lineiii{\member{preinst}}{\member{string}}{Script to be executed before package installation.}
  \lineiii{\member{postinst}}{\member{string}}{Script to be executed after package installation.}
  \lineiii{\member{prerm}}{\member{string}}{Script to be executed before package removal.}
  \lineiii{\member{postrm}}{\member{string}}{Script to be executed after package removal.}
\end{tableiii}
\citetitle{Optional Variables}

\index{debinfo example}
Example:
\begin{verbatim}
fetch_method = "svn"
svn_module = "system"

control = {
	'Package': 'python-skolesys-client',
	'Version': 'file://skolesys_ver',
	'NameExtension': 'feisty_all',
	'Section': 'python',
	'Priority': 'optional',
	'Architecture': 'all',
	'Depends': 'python-support (>= 0.2), python-soappy, python-m2crypto',
	'Recommends': 'skolesys-qt4',
	'Maintainer': 'Jakob Simon-Gaarde <jakob@skolesys.dk>',
	'Replaces': 'python-skolesys-seeder, python2.4-skolesys-seeder, python2.4-skolesys-client',
	'Conflicts': 'python2.4-skolesys-mainserver, python2.4-skolesys-seeder, python-skolesys-mainserver, python-skolesys-seeder',
	'Provides': 'python2.5-skolesys-client, python2.3-skolesys-client, python2.4-skolesys-client',
	'Description': 'This is the soap client part of the SkoleSYS linux distribution',
	'longdesc': 
""" The skolesys package provides the nessecary tools for administrating the SkoleSYS
 distribution. The main issue here is creating users and groups, controlling permissions,
 creating user and group spaces, registering client workstations (Windows, Linux, MacOS)
 and registering thin client servers (LTSP).
"""}

perm = {'cfmachine/cfinstaller.py': '755',
	'soap/getconf.py': '755',
	'soap/reghost.py': '755'}

copy = {
	'__init__.py': '/usr/share/python-support/python-skolesys-client/skolesys/',
	'soap/__init__.py': '/usr/share/python-support/python-skolesys-client/skolesys/soap',
	'soap/netinfo.py': '/usr/share/python-support/python-skolesys-client/skolesys/soap',
	'soap/marshall.py': '/usr/share/python-support/python-skolesys-client/skolesys/soap',
	'soap/getconf.py': '/usr/share/python-support/python-skolesys-client/skolesys/soap',
	'soap/reghost.py': '/usr/share/python-support/python-skolesys-client/skolesys/soap',
	'soap/client.py': '/usr/share/python-support/python-skolesys-client/skolesys/soap',
	'soap/p2.py': '/usr/share/python-support/python-skolesys-client/skolesys/soap',
	'cfmachine/__init__.py': '/usr/share/python-support/python-skolesys-client/skolesys/cfmachine',
	'cfmachine/cfinstaller.py': '/usr/share/python-support/python-skolesys-client/skolesys/cfmachine',
	'cfmachine/apthelpers.py': '/usr/share/python-support/python-skolesys-client/skolesys/cfmachine',
	'cfmachine/fstabhelpers.py': '/usr/share/python-support/python-skolesys-client/skolesys/cfmachine',
	'tools': '/usr/share/python-support/python-skolesys-client/skolesys/',
	'definitions': '/usr/share/python-support/python-skolesys-client/skolesys/'}

links = {
	'/usr/sbin/ss_installer': '../share/python-support/python-skolesys-client/skolesys/cfmachine/cfinstaller.py',
	'/usr/sbin/ss_getconf': '../share/python-support/python-skolesys-client/skolesys/soap/getconf.py',
	'/usr/sbin/ss_reghost': '../share/python-support/python-skolesys-client/skolesys/soap/reghost.py'}

postinst = """#!/bin/sh
set -e
# Automatically added by dh_pysupport
if [ "$1" = "configure" ] && which update-python-modules >/dev/null 2>&1; then
        update-python-modules -i /usr/share/python-support/python-skolesys-client
fi
# End automatically added section
"""

prerm = """#!/bin/sh
set -e
# Automatically added by dh_pysupport
if which update-python-modules >/dev/null 2>&1; then
        update-python-modules -c -i /usr/share/python-support/python-skolesys-client
fi
# End automatically added section
"""

postrm = """#!/bin/sh
if [ -e /usr/share/python-support/python-skolesys-client/skolesys ]
then
  find /usr/share/python-support/python-skolesys-client/skolesys -name "*.pyc" -delete
  find /usr/share/python-support/python-skolesys-client/skolesys -name "*.pyo" -delete
fi
"""
\end{verbatim}

\subsection{Subversion based modules\label{skolesys-deb-svn}}
Most of the skolesys-deb modules rely on fetching their content from a subversion repository. It is possible to setup this repository once and for all by setting the SKOLESYS_SVNBASE environment variable. 

However, if you need to fetch content for different skolesys-deb modules from different repositories you can specify the repository for each module by setting the svn_repos variable in the debinfo file or you can do it by using the -r command line option. svn_repos will override SKOLESYS_SVNBASE, -r command line options overrides everything.
The debinfo variable svn_module specifies the path to fetch from inside the repository. If svn_module is unset skolesys-deb will try to fetch using the module_name as path.

\begin{tableii}{c|l}{}{Environment Variable}{Description}
  \lineii{\member{SKOLESYS_SVNBASE}}{The default subversion repository or repository parent dir.}
\end{tableii}

\note{If you need to fetch from the base of a repository you will have to set svn_repos to the parent dir of the repository - so if I need to fetch from the root of the repository located at svn.mydomain.org/srv/svn/skolesys I will need to set svn_repos to somthing like svn+ssh://svn.mydomain.org/srv/svn/ and svn_module to skolesys.}

\subsection{ISO and tar.gz based modules\label{skolesys-deb-iso}}
There is not much to say about these fetch methods. All skolesys-deb needs to know is where to find the iso or tgz files. Like when using subversion for fetch method you can set environment variables that will serve as default directories for iso/tgz file hunting.

command line options -l iso_dir and -t  tgz_dir can also be used to set the directory location and these will of course override the environment variables if set. 

\begin{tableii}{c|l}{}{Environment Variable}{Description}
  \lineii{\member{SKOLESYS_ISODIR}}{The default directory when searching for ISO files.}
  \lineii{\member{SKOLESYS_TGZDIR}}{The default directory when searching for tgz files.}
\end{tableii}

\section{skolesys-apt\label{skolesys-apt}}
After creating debian packages the next thing needed is to make life easier for debian users who want to use the packages. To do that we are going to create a standard debian package repository which debian APT (Advanced Package Tool) relies on. 

skolesys-apt can do this automatically, all we need to do is create a simple aptinfo file and place the debian files in a distribution hierarchi - also we need a gpg key-pair to sign the APT Release file. If you don't have such a key-pair it will be shown how to gererate it.

skolesys-apt will create the APT repository on the local filesystem ready to ship by copying the two folders \member{dists} and \member{pool} and the public part of the gpg key-pair to the host running the webserver that is going to expose the repository.

\index{skolesys-apt}
\member{Usage: skolesys-apt dist_name}

\subsection{Prepair gpg key-pair\label{skolesys-apt-gpg}}

Prerequisite: You need \member{gnupg} installed to go any further.

\subsubsection{Reuse an existing gpg key-pair\label{skolesys-apt-reusegpg}}
If you already have a gpg key-pair you wish to use for signing the APT repository it os possible to export the key-pair and then import it on the host where you are running skolesys-apt.
Let us say that the user gpg-owner on host gpg-source-host has the key-pair we want to export and the user skolesys-devel on host gpg-dest-host needs to import it:

Export key-pair:
\begin{enumerate}
 \item log on to the source host gpg-owner@gpg-source-host
 \item Export the gpg key-pair:
\begin{verbatim}
gpg-owner@gpg-source-host$ gpg --armor --export-secret-keys EEF2B7FA > secret.gpg.asc
gpg-owner@gpg-source-host$ gpg --armor --export EEF2B7FA > public.gpg.asc
\end{verbatim}
\end{enumerate}
\note{EEF2B7FA is the key ID. Use gpg --list-keys to view your key ID's}

Import key-pair:

\begin{enumerate}
 \item Copy the secret.gpg.asc and public.gpg.asc from the source host
 \item log on to the destination host skolesys-developer@gpg-dest-host
 \item Import the key-pair
\begin{verbatim}
skolesys-devel@gpg-dest-host$ gpg --allow-secret-key-import --import public.gpg.asc secret.gpg.asc
\end{verbatim}
\end{enumerate}

\member{public.gpg.asc} will be the public key block that you must expose to the public to make your signed APT repository validate and thereby accessible to remote debian based systems.

\subsubsection{Create a gpg key-pair\label{skolesys-apt-creategpg}}
If you don't have a gpg key-pair to use for this task then you need to create one now.
\begin{enumerate}
 \item Create the gpg key-pair
 \begin{verbatim}
gpg --gen-key

.... (I used the no-existing email: jakob@email.com)

pub   1024D/55117311 2007-06-21
	Key fingerprint = F461 E5AF 0DE8 B3E4 358D  DEB6 EC23 7CF4 5511 7311
uid                  Jakob Simon-Gaarde <jakob@email.com>
sub   2048g/FDD61F6E 2007-06-21
 \end{verbatim}
 \note{It is not required that you supply a passphrase you can leave it empty.}

 \item In the example above the newly generated key-pair has ID 55117311
 \item Extract a public key block	
 \begin{verbatim}
gpg --export --armor jakob@email.com > public.gpg.asc
 \end{verbatim}

 \member{public.gpg.asc} will be the public key block that you must expose to the public to make your signed APT repository validate and thereby accessible to remote debian based systems.

\end{enumerate}

\subsection{Creating a distribution hierarchi\label{skolesys-apt-pool}}
Most of the skolesys-deb modules rely on fetching their content from a subversion repository. It is possible to setup this repository once and for all by setting the SKOLESYS_SVNBASE environment variable. 

However, if you need to fetch content for different skolesys-deb modules from different repositories you can specify the repository for each module by setting the svn_repos variable in the debinfo file or you can do it by using the -r command line option. svn_repos will override SKOLESYS_SVNBASE, -r command line options overrides everything.
The debinfo variable svn_module specifies the path to fetch from inside the repository. If svn_module is unset skolesys-deb will try to fetch using the module_name as path.

\begin{tableii}{c|l}{}{Environment Variable}{Description}
  \lineii{\member{SKOLESYS_SVNBASE}}{The default subversion repository or repository parent dir.}
\end{tableii}

\note{If you need to fetch from the base of a repository you will have to set svn_repos to the parent dir of the repository - so if I need to fetch from the root of the repository located at svn.mydomain.org/srv/svn/skolesys I will need to set svn_repos to somthing like svn+ssh://svn.mydomain.org/srv/svn/ and svn_module to skolesys.}

\subsection{Creating aptinfo file\label{skolesys-apt-aptinfo}}
There is not much to say about these fetch methods. All skolesys-deb needs to know is where to find the iso or tgz files. Like when using subversion for fetch method you can set environment variables that will serve as default directories for iso/tgz file hunting.

command line options -l iso_dir and -t  tgz_dir can also be used to set the directory location and these will of course override the environment variables if set. 

\begin{tableii}{c|l}{}{Environment Variable}{Description}
  \lineii{\member{SKOLESYS_ISODIR}}{The default directory when searching for ISO files.}
  \lineii{\member{SKOLESYS_TGZDIR}}{The default directory when searching for tgz files.}
\end{tableii}



\documentclass{manual}

\title{SkoleSYS Developer Documentation}

\makeindex			% tell \index to actually write the .idx file


\begin{document}

\maketitle

\ifhtml
\chapter*{Front Matter\label{front}}
\fi

\begin{abstract}

\noindent
This document addresses the developer specific issues of SkoleSYS. It is
split up in two main parts - SkoleSYS library documentation and SkoleSYS
distribution tools (disttools).

SkoleSYS Library Documentation coveres the skolesys python package wich
is essentially all SkoleSYS logic e.g. client, server, configuration, SOAP
and more.

SkoleSYS Distribution Tools documents the three tools for deb-packaging,
APT distribution and install CD creation (skolesys-deb, skolesys-apt and 
skolesys-cd)

\end{abstract}

\tableofcontents

\chapter{SkoleSYS Distribution Tools (trunk/disttools)\label{disttools}}
The SkoleSYS Distribution Tools are just convenience tools I have made to make the job of distributing SkoleSYS fast and easy. That way I can keep the main focus on developing the SkoleSYS libraries and GUI. 

The \member{skolesys-apt} script can fetch content via subversion, tarballs or iso-files and place data where you want it redistributed on target filesystems. You can define symlinks and install-scripts (preinst, postinst, prerm, postrm) and setup file permissions. It is also possible to register a scripts to be run by \member{skolesys-apt} after content extraction (ie. pre-bytecompile python scripts). Once the debinfo files (control files for \member{skolesys-apt}) are in place building a file is as easy as executing one command - you can even setup automatic version ticking and group debian packages together by syncronizing their version number.

After building debian packages \member{skolesys-apt} is the tool for building an APT archive. You only need to setup one aptinfo file (control file for \member{skolesys-apt}) per distribution you target create a \member{distribution hieararchi}\footnote{A distribution hieararchi tells \member{skolesys-apt} which packages go into which distribution components and which computer architecture they are build for.}. \member{skolesys-apt} will automatically generate the \member{dists} and \member{pool} directory so they are ready to put on your distribution webserver.
\section{skolesys-deb\label{skolesys-deb}}
This tool is used to create deb-files for distribution. It can fetch distribution content using svn, iso or tgz. Each deb-file is controlled by a debinfo file, the naming convention for such a file is \member{<module_name>_debinfo.py} - e.g. the module named \member{"python-skolesys-client_feisty"} should be named: \member{"python-skolesys-client_feisty_debinfo.py"}. Module name should not be confused with package name, the package name is specified inside the debinfo file.

\index{skolesys-deb}
\member{Usage: skolesys-deb [options] module_name}


\begin{tableiii}{l|l|l}{command}{Short-option}{Long-option}{Description}
  \lineiii{-r \var{url}}{\member{--svnbase=\var{url}}}{URL to the subversion repository to fetch from (use when \member{fetch_method='svn'})}
  \lineiii{\member{-l \var{dir}}}{\member{--iso-location=\var{dir}}}{The directory containing the iso file (use when \member{fetch_method='iso'})}
  \lineiii{\member{-l \var{dir}}}{\member{--iso-location=\var{dir}}}{The directory containing the tar.gz file (use when \member{fetch_method='tgz'})}
  \lineiii{}{\member{--dont-fetch}}{Don't fetch the package contents. This is used if more than one package is based on the same content resource, so if these packages are build successively there is no need to fetch the content more than once.}
  \lineiii{\member{-h}}{\member{--help}}{Show the help message and exit}
\end{tableiii}
\citetitle{skolesys-deb command line options}
\subsection{debinfo files\label{debinfo}}
Describing the naming convention has already revealed that the python interpreter is used for parsing debinfo files, and that is also why I say module name instead of package name. A debinfo file consists of a number of global variables, most of these are optional but there are three mandatory:

\index{debinfo file options}
\begin{tableiii}{c|l|l}{textrm}{Name}{Type}{Description}
  \lineiii{\member{fetch_method}}{\member{string}}{Specify how to fetch the content for the deb-package.}
  \lineiii{\member{control}}{\member{dict}}{This dictionary holds the normal dpkg control structure. There is a special feature regarding the Version value for grouping packages together giving them the same package version. Simply set the value to point to the file containing the version number in the first line. (e.g. 'Version': 'file://ver_no')}
  \lineiii{\member{copy}}{\member{dict}}{Specify which files should go where in the filesystem during installation.}
\end{tableiii}
\citetitle{Mandatory variables}

\begin{tableiii}{c|l|l}{textrm}{Name}{Type}{Description}
  \lineiii{\member{svn_module}}{\member{string}}{When using subversion as fetch method you can use this variable to specify a certain path in the repository. If this variable is not set skolesys-deb will try to use the module_name as path.}
  \lineiii{\member{svn_repos}}{\member{string}}{Specify the subversion repository to fetch the package source from. If this variable is set it will override the SKOLESYS_SVNBASE environment variable but not the command line argument -r.}
  \lineiii{\member{prebuild_script}}{\member{string}}{Script to be executed after content extraction from svn, iso or tgz and before the deb-file creation (see \member{skolesys-qt3_dapper_debinfo.py}) }
  \lineiii{\member{links}}{\member{dict}}{Each key-value entry in the links dict specifies a symbolic link to be created in the filesystem after installation.}
  \lineiii{\member{perm}}{\member{dict}}{The perm variable can be used to setup file permissions for the files being installed.}
  \lineiii{\member{preinst}}{\member{string}}{Script to be executed before package installation.}
  \lineiii{\member{postinst}}{\member{string}}{Script to be executed after package installation.}
  \lineiii{\member{prerm}}{\member{string}}{Script to be executed before package removal.}
  \lineiii{\member{postrm}}{\member{string}}{Script to be executed after package removal.}
\end{tableiii}
\citetitle{Optional Variables}

\index{debinfo example}
Example:
\begin{verbatim}
fetch_method = "svn"
svn_module = "system"

control = {
	'Package': 'python-skolesys-client',
	'Version': 'file://skolesys_ver',
	'NameExtension': 'feisty_all',
	'Section': 'python',
	'Priority': 'optional',
	'Architecture': 'all',
	'Depends': 'python-support (>= 0.2), python-soappy, python-m2crypto',
	'Recommends': 'skolesys-qt4',
	'Maintainer': 'Jakob Simon-Gaarde <jakob@skolesys.dk>',
	'Replaces': 'python-skolesys-seeder, python2.4-skolesys-seeder, python2.4-skolesys-client',
	'Conflicts': 'python2.4-skolesys-mainserver, python2.4-skolesys-seeder, python-skolesys-mainserver, python-skolesys-seeder',
	'Provides': 'python2.5-skolesys-client, python2.3-skolesys-client, python2.4-skolesys-client',
	'Description': 'This is the soap client part of the SkoleSYS linux distribution',
	'longdesc': 
""" The skolesys package provides the nessecary tools for administrating the SkoleSYS
 distribution. The main issue here is creating users and groups, controlling permissions,
 creating user and group spaces, registering client workstations (Windows, Linux, MacOS)
 and registering thin client servers (LTSP).
"""}

perm = {'cfmachine/cfinstaller.py': '755',
	'soap/getconf.py': '755',
	'soap/reghost.py': '755'}

copy = {
	'__init__.py': '/usr/share/python-support/python-skolesys-client/skolesys/',
	'soap/__init__.py': '/usr/share/python-support/python-skolesys-client/skolesys/soap',
	'soap/netinfo.py': '/usr/share/python-support/python-skolesys-client/skolesys/soap',
	'soap/marshall.py': '/usr/share/python-support/python-skolesys-client/skolesys/soap',
	'soap/getconf.py': '/usr/share/python-support/python-skolesys-client/skolesys/soap',
	'soap/reghost.py': '/usr/share/python-support/python-skolesys-client/skolesys/soap',
	'soap/client.py': '/usr/share/python-support/python-skolesys-client/skolesys/soap',
	'soap/p2.py': '/usr/share/python-support/python-skolesys-client/skolesys/soap',
	'cfmachine/__init__.py': '/usr/share/python-support/python-skolesys-client/skolesys/cfmachine',
	'cfmachine/cfinstaller.py': '/usr/share/python-support/python-skolesys-client/skolesys/cfmachine',
	'cfmachine/apthelpers.py': '/usr/share/python-support/python-skolesys-client/skolesys/cfmachine',
	'cfmachine/fstabhelpers.py': '/usr/share/python-support/python-skolesys-client/skolesys/cfmachine',
	'tools': '/usr/share/python-support/python-skolesys-client/skolesys/',
	'definitions': '/usr/share/python-support/python-skolesys-client/skolesys/'}

links = {
	'/usr/sbin/ss_installer': '../share/python-support/python-skolesys-client/skolesys/cfmachine/cfinstaller.py',
	'/usr/sbin/ss_getconf': '../share/python-support/python-skolesys-client/skolesys/soap/getconf.py',
	'/usr/sbin/ss_reghost': '../share/python-support/python-skolesys-client/skolesys/soap/reghost.py'}

postinst = """#!/bin/sh
set -e
# Automatically added by dh_pysupport
if [ "$1" = "configure" ] && which update-python-modules >/dev/null 2>&1; then
        update-python-modules -i /usr/share/python-support/python-skolesys-client
fi
# End automatically added section
"""

prerm = """#!/bin/sh
set -e
# Automatically added by dh_pysupport
if which update-python-modules >/dev/null 2>&1; then
        update-python-modules -c -i /usr/share/python-support/python-skolesys-client
fi
# End automatically added section
"""

postrm = """#!/bin/sh
if [ -e /usr/share/python-support/python-skolesys-client/skolesys ]
then
  find /usr/share/python-support/python-skolesys-client/skolesys -name "*.pyc" -delete
  find /usr/share/python-support/python-skolesys-client/skolesys -name "*.pyo" -delete
fi
"""
\end{verbatim}

\subsection{Subversion based modules\label{skolesys-deb-svn}}
Most of the skolesys-deb modules rely on fetching their content from a subversion repository. It is possible to setup this repository once and for all by setting the SKOLESYS_SVNBASE environment variable. 

However, if you need to fetch content for different skolesys-deb modules from different repositories you can specify the repository for each module by setting the svn_repos variable in the debinfo file or you can do it by using the -r command line option. svn_repos will override SKOLESYS_SVNBASE, -r command line options overrides everything.
The debinfo variable svn_module specifies the path to fetch from inside the repository. If svn_module is unset skolesys-deb will try to fetch using the module_name as path.

\begin{tableii}{c|l}{}{Environment Variable}{Description}
  \lineii{\member{SKOLESYS_SVNBASE}}{The default subversion repository or repository parent dir.}
\end{tableii}

\note{If you need to fetch from the base of a repository you will have to set svn_repos to the parent dir of the repository - so if I need to fetch from the root of the repository located at svn.mydomain.org/srv/svn/skolesys I will need to set svn_repos to somthing like svn+ssh://svn.mydomain.org/srv/svn/ and svn_module to skolesys.}

\subsection{ISO and tar.gz based modules\label{skolesys-deb-iso}}
There is not much to say about these fetch methods. All skolesys-deb needs to know is where to find the iso or tgz files. Like when using subversion for fetch method you can set environment variables that will serve as default directories for iso/tgz file hunting.

command line options -l iso_dir and -t  tgz_dir can also be used to set the directory location and these will of course override the environment variables if set. 

\begin{tableii}{c|l}{}{Environment Variable}{Description}
  \lineii{\member{SKOLESYS_ISODIR}}{The default directory when searching for ISO files.}
  \lineii{\member{SKOLESYS_TGZDIR}}{The default directory when searching for tgz files.}
\end{tableii}

\section{skolesys-apt\label{skolesys-apt}}
After creating debian packages the next thing needed is to make life easier for debian users who want to use the packages. To do that we are going to create a standard debian package repository which debian APT (Advanced Package Tool) relies on. 

skolesys-apt can do this automatically, all we need to do is create a simple aptinfo file and place the debian files in a distribution hierarchi - also we need a gpg key-pair to sign the APT Release file. If you don't have such a key-pair it will be shown how to gererate it.

skolesys-apt will create the APT repository on the local filesystem ready to ship by copying the two folders \member{dists} and \member{pool} and the public part of the gpg key-pair to the host running the webserver that is going to expose the repository.

\index{skolesys-apt}
\member{Usage: skolesys-apt dist_name}

\subsection{Prepair gpg key-pair\label{skolesys-apt-gpg}}

Prerequisite: You need \member{gnupg} installed to go any further.

\subsubsection{Reuse an existing gpg key-pair\label{skolesys-apt-reusegpg}}
If you already have a gpg key-pair you wish to use for signing the APT repository it os possible to export the key-pair and then import it on the host where you are running skolesys-apt.
Let us say that the user gpg-owner on host gpg-source-host has the key-pair we want to export and the user skolesys-devel on host gpg-dest-host needs to import it:

Export key-pair:
\begin{enumerate}
 \item log on to the source host gpg-owner@gpg-source-host
 \item Export the gpg key-pair:
\begin{verbatim}
gpg-owner@gpg-source-host$ gpg --armor --export-secret-keys EEF2B7FA > secret.gpg.asc
gpg-owner@gpg-source-host$ gpg --armor --export EEF2B7FA > public.gpg.asc
\end{verbatim}
\end{enumerate}
\note{EEF2B7FA is the key ID. Use gpg --list-keys to view your key ID's}

Import key-pair:

\begin{enumerate}
 \item Copy the secret.gpg.asc and public.gpg.asc from the source host
 \item log on to the destination host skolesys-developer@gpg-dest-host
 \item Import the key-pair
\begin{verbatim}
skolesys-devel@gpg-dest-host$ gpg --allow-secret-key-import --import public.gpg.asc secret.gpg.asc
\end{verbatim}
\end{enumerate}

\member{public.gpg.asc} will be the public key block that you must expose to the public to make your signed APT repository validate and thereby accessible to remote debian based systems.

\subsubsection{Create a gpg key-pair\label{skolesys-apt-creategpg}}
If you don't have a gpg key-pair to use for this task then you need to create one now.
\begin{enumerate}
 \item Create the gpg key-pair
 \begin{verbatim}
gpg --gen-key

.... (I used the no-existing email: jakob@email.com)

pub   1024D/55117311 2007-06-21
	Key fingerprint = F461 E5AF 0DE8 B3E4 358D  DEB6 EC23 7CF4 5511 7311
uid                  Jakob Simon-Gaarde <jakob@email.com>
sub   2048g/FDD61F6E 2007-06-21
 \end{verbatim}
 \note{It is not required that you supply a passphrase you can leave it empty.}

 \item In the example above the newly generated key-pair has ID 55117311
 \item Extract a public key block	
 \begin{verbatim}
gpg --export --armor jakob@email.com > public.gpg.asc
 \end{verbatim}

 \member{public.gpg.asc} will be the public key block that you must expose to the public to make your signed APT repository validate and thereby accessible to remote debian based systems.

\end{enumerate}

\subsection{Creating a distribution hierarchi\label{skolesys-apt-pool}}
Most of the skolesys-deb modules rely on fetching their content from a subversion repository. It is possible to setup this repository once and for all by setting the SKOLESYS_SVNBASE environment variable. 

However, if you need to fetch content for different skolesys-deb modules from different repositories you can specify the repository for each module by setting the svn_repos variable in the debinfo file or you can do it by using the -r command line option. svn_repos will override SKOLESYS_SVNBASE, -r command line options overrides everything.
The debinfo variable svn_module specifies the path to fetch from inside the repository. If svn_module is unset skolesys-deb will try to fetch using the module_name as path.

\begin{tableii}{c|l}{}{Environment Variable}{Description}
  \lineii{\member{SKOLESYS_SVNBASE}}{The default subversion repository or repository parent dir.}
\end{tableii}

\note{If you need to fetch from the base of a repository you will have to set svn_repos to the parent dir of the repository - so if I need to fetch from the root of the repository located at svn.mydomain.org/srv/svn/skolesys I will need to set svn_repos to somthing like svn+ssh://svn.mydomain.org/srv/svn/ and svn_module to skolesys.}

\subsection{Creating aptinfo file\label{skolesys-apt-aptinfo}}
There is not much to say about these fetch methods. All skolesys-deb needs to know is where to find the iso or tgz files. Like when using subversion for fetch method you can set environment variables that will serve as default directories for iso/tgz file hunting.

command line options -l iso_dir and -t  tgz_dir can also be used to set the directory location and these will of course override the environment variables if set. 

\begin{tableii}{c|l}{}{Environment Variable}{Description}
  \lineii{\member{SKOLESYS_ISODIR}}{The default directory when searching for ISO files.}
  \lineii{\member{SKOLESYS_TGZDIR}}{The default directory when searching for tgz files.}
\end{tableii}

%\section{skolesys-cd\label{skolesys-cd}}
This tool is used to create deb-files for distribution. It can fetch distribution content using svn, iso or tgz. Each deb-file is controlled by a debinfo file, the naming convention for such a file is \member{<module_name>_debinfo.py} - e.g. the module named \member{"python-skolesys-client_feisty"} should be named: \member{"python-skolesys-client_feisty_debinfo.py"}. Module name should not be confused with package name, the package name is specified inside the debinfo file.

\begin{verbatim}
skolesys-deb[-r svn_repos][-l iso_dir][-t tgz_dir][--dont-fetch] module_name
\end{verbatim}


\begin{tableiii}{l|l|l}{command}{Short-option}{Long-option}{Description}
  \lineiii{-r \var{url}}{\member{--svnbase=\var{url}}}{URL to the subversion repository to fetch from (use when \member{fetch_method='svn'})}
  \lineiii{\member{-l \var{dir}}}{\member{--iso-location=\var{dir}}}{The directory containing the iso file (use when \member{fetch_method='iso'})}
  \lineiii{\member{-l \var{dir}}}{\member{--iso-location=\var{dir}}}{The directory containing the tar.gz file (use when \member{fetch_method='tgz'})}
  \lineiii{}{\member{--dont-fetch}}{Don't fetch the package contents. This is used if more than one package is based on the same content resource, so if these packages are build successively there is no need to fetch the content more than once.}
  \lineiii{\member{-h}}{\member{--help}}{Show the help message and exit}
\end{tableiii}

\subsection{debinfo files\label{debinfo}}
Describing the naming convention has already revealed that the python interpreter is used for parsing debinfo files, and that is also why I say module name instead of package name. A debinfo file consists of a number of global variables, most of these are optional but there are three mandatory.

Mandatory variables:
\begin{tableiii}{c|l|l}{textrm}{Name}{Type}{Description}
  \lineiii{\member{fetch_method}}{\member{string}}{Specify how to fetch the content for the deb-package.}
  \lineiii{\member{control}}{\member{dict}}{This dictionary holds the normal dpkg control structure. There is a special feature regarding the Version value for grouping packages together giving them the same package version. Simply set the value to point to the file containing the version number in the first line. (e.g. 'Version': 'file://ver_no')}
  \lineiii{\member{copy}}{\member{dict}}{Specify which files should go where in the filesystem during installation.}
\end{tableiii}

Optional Variables:
\begin{tableiii}{c|l|l}{textrm}{Name}{Type}{Description}
  \lineiii{\member{svn_module}}{\member{string}}{When using subversion as fetch method you can use this variable to specify a certain path in the repository. If this variable is not set skolesys-deb will try to use the module_name as path.}
  \lineiii{\member{svn_repos}}{\member{string}}{Specify the subversion repository to fetch the package source from. If this variable is set it will override the SKOLESYS_SVNBASE environment variable but not the command line argument -r.}
  \lineiii{\member{links}}{\member{dict}}{Each key-value entry in the links dict specifies a symbolic link to be created in the filesystem after installation.}
  \lineiii{\member{perm}}{\member{dict}}{The perm variable can be used to setup file permissions for the files being installed.}
  \lineiii{\member{preinst}}{\member{string}}{Script to be executed before package installation.}
  \lineiii{\member{postinst}}{\member{string}}{Script to be executed after package installation.}
  \lineiii{\member{prerm}}{\member{string}}{Script to be executed before package removal.}
  \lineiii{\member{postrm}}{\member{string}}{Script to be executed after package removal.}
\end{tableiii}

Example:
\begin{verbatim}
fetch_method = "svn"
svn_module = "system"

control = {
	'Package': 'python-skolesys-client',
	'Version': 'file://skolesys_ver',
	'NameExtension': 'feisty_all',
	'Section': 'python',
	'Priority': 'optional',
	'Architecture': 'all',
	'Depends': 'python-support (>= 0.2), python-soappy, python-m2crypto',
	'Recommends': 'skolesys-qt4',
	'Maintainer': 'Jakob Simon-Gaarde <jakob@skolesys.dk>',
	'Replaces': 'python-skolesys-seeder, python2.4-skolesys-seeder, python2.4-skolesys-client',
	'Conflicts': 'python2.4-skolesys-mainserver, python2.4-skolesys-seeder, python-skolesys-mainserver, python-skolesys-seeder',
	'Provides': 'python2.5-skolesys-client, python2.3-skolesys-client, python2.4-skolesys-client',
	'Description': 'This is the soap client part of the SkoleSYS linux distribution',
	'longdesc': 
""" The skolesys package provides the nessecary tools for administrating the SkoleSYS
 distribution. The main issue here is creating users and groups, controlling permissions,
 creating user and group spaces, registering client workstations (Windows, Linux, MacOS)
 and registering thin client servers (LTSP).
"""}

perm = {'cfmachine/cfinstaller.py': '755',
	'soap/getconf.py': '755',
	'soap/reghost.py': '755'}

copy = {
	'__init__.py': '/usr/share/python-support/python-skolesys-client/skolesys/',
	'soap/__init__.py': '/usr/share/python-support/python-skolesys-client/skolesys/soap',
	'soap/netinfo.py': '/usr/share/python-support/python-skolesys-client/skolesys/soap',
	'soap/marshall.py': '/usr/share/python-support/python-skolesys-client/skolesys/soap',
	'soap/getconf.py': '/usr/share/python-support/python-skolesys-client/skolesys/soap',
	'soap/reghost.py': '/usr/share/python-support/python-skolesys-client/skolesys/soap',
	'soap/client.py': '/usr/share/python-support/python-skolesys-client/skolesys/soap',
	'soap/p2.py': '/usr/share/python-support/python-skolesys-client/skolesys/soap',
	'cfmachine/__init__.py': '/usr/share/python-support/python-skolesys-client/skolesys/cfmachine',
	'cfmachine/cfinstaller.py': '/usr/share/python-support/python-skolesys-client/skolesys/cfmachine',
	'cfmachine/apthelpers.py': '/usr/share/python-support/python-skolesys-client/skolesys/cfmachine',
	'cfmachine/fstabhelpers.py': '/usr/share/python-support/python-skolesys-client/skolesys/cfmachine',
	'tools': '/usr/share/python-support/python-skolesys-client/skolesys/',
	'definitions': '/usr/share/python-support/python-skolesys-client/skolesys/'}

links = {
	'/usr/sbin/ss_installer': '../share/python-support/python-skolesys-client/skolesys/cfmachine/cfinstaller.py',
	'/usr/sbin/ss_getconf': '../share/python-support/python-skolesys-client/skolesys/soap/getconf.py',
	'/usr/sbin/ss_reghost': '../share/python-support/python-skolesys-client/skolesys/soap/reghost.py'}

postinst = """#!/bin/sh
set -e
# Automatically added by dh_pysupport
if [ "$1" = "configure" ] && which update-python-modules >/dev/null 2>&1; then
        update-python-modules -i /usr/share/python-support/python-skolesys-client
fi
# End automatically added section
"""

prerm = """#!/bin/sh
set -e
# Automatically added by dh_pysupport
if which update-python-modules >/dev/null 2>&1; then
        update-python-modules -c -i /usr/share/python-support/python-skolesys-client
fi
# End automatically added section
"""

postrm = """#!/bin/sh
if [ -e /usr/share/python-support/python-skolesys-client/skolesys ]
then
  find /usr/share/python-support/python-skolesys-client/skolesys -name "*.pyc" -delete
  find /usr/share/python-support/python-skolesys-client/skolesys -name "*.pyo" -delete
fi
"""
\end{verbatim}

\subsection{Subversion based modules\label{skolesys-deb-svn}}
Most of the skolesys-deb modules rely on fetching their content from a subversion repository. It is possible to setup this repository once and for all by setting the SKOLESYS_SVNBASE environment variable. 

However, if you need to fetch content for different skolesys-deb modules from different repositories you can specify the repository for each module by setting the svn_repos variable in the debinfo file or you can do it by using the -r command line option. svn_repos will override SKOLESYS_SVNBASE, -r command line options overrides everything.
The debinfo variable svn_module specifies the path to fetch from inside the repository. If svn_module is unset skolesys-deb will try to fetch using the module_name as path.

\begin{tableii}{c|l}{}{Environment Variable}{Description}
  \lineii{\member{SKOLESYS_SVNBASE}}{The default subversion repository or repository parent dir.}
\end{tableii}

\note{If you need to fetch from the base of a repository you will have to set svn_repos to the parent dir of the repository - so if I need to fetch from the root of the repository located at svn.mydomain.org/srv/svn/skolesys I will need to set svn_repos to somthing like svn+ssh://svn.mydomain.org/srv/svn/ and svn_module to skolesys.}

\subsection{ISO and tar.gz based modules\label{skolesys-deb-iso}}
There is not much to say about these fetch methods. All skolesys-deb needs to know is where to find the iso or tgz files. Like when using subversion for fetch method you can set environment variables that will serve as default directories for iso/tgz file hunting.

command line options -l iso_dir and -t  tgz_dir can also be used to set the directory location and these will of course override the environment variables if set. 

\begin{tableii}{c|l}{}{Environment Variable}{Description}
  \lineii{\member{SKOLESYS_ISODIR}}{The default directory when searching for ISO files.}
  \lineii{\member{SKOLESYS_TGZDIR}}{The default directory when searching for tgz files.}
\end{tableii}



% \chapter{Debugging \label{debugging}}
%
% XXX Explain Py_DEBUG, Py_TRACE_REFS, Py_REF_DEBUG.

\chapter{SkoleSYS Package (trunk/system)\label{skolesys-pack}}
\section{skolesys-deb\label{skolesys-deb}}
This tool is used to create deb-files for distribution. It can fetch distribution content using svn, iso or tgz. Each deb-file is controlled by a debinfo file, the naming convention for such a file is \member{<module_name>_debinfo.py} - e.g. the module named \member{"python-skolesys-client_feisty"} should be named: \member{"python-skolesys-client_feisty_debinfo.py"}. Module name should not be confused with package name, the package name is specified inside the debinfo file.

\index{skolesys-deb}
\member{Usage: skolesys-deb [options] module_name}


\begin{tableiii}{l|l|l}{command}{Short-option}{Long-option}{Description}
  \lineiii{-r \var{url}}{\member{--svnbase=\var{url}}}{URL to the subversion repository to fetch from (use when \member{fetch_method='svn'})}
  \lineiii{\member{-l \var{dir}}}{\member{--iso-location=\var{dir}}}{The directory containing the iso file (use when \member{fetch_method='iso'})}
  \lineiii{\member{-l \var{dir}}}{\member{--iso-location=\var{dir}}}{The directory containing the tar.gz file (use when \member{fetch_method='tgz'})}
  \lineiii{}{\member{--dont-fetch}}{Don't fetch the package contents. This is used if more than one package is based on the same content resource, so if these packages are build successively there is no need to fetch the content more than once.}
  \lineiii{\member{-h}}{\member{--help}}{Show the help message and exit}
\end{tableiii}
\citetitle{skolesys-deb command line options}
\subsection{debinfo files\label{debinfo}}
Describing the naming convention has already revealed that the python interpreter is used for parsing debinfo files, and that is also why I say module name instead of package name. A debinfo file consists of a number of global variables, most of these are optional but there are three mandatory:

\index{debinfo file options}
\begin{tableiii}{c|l|l}{textrm}{Name}{Type}{Description}
  \lineiii{\member{fetch_method}}{\member{string}}{Specify how to fetch the content for the deb-package.}
  \lineiii{\member{control}}{\member{dict}}{This dictionary holds the normal dpkg control structure. There is a special feature regarding the Version value for grouping packages together giving them the same package version. Simply set the value to point to the file containing the version number in the first line. (e.g. 'Version': 'file://ver_no')}
  \lineiii{\member{copy}}{\member{dict}}{Specify which files should go where in the filesystem during installation.}
\end{tableiii}
\citetitle{Mandatory variables}

\begin{tableiii}{c|l|l}{textrm}{Name}{Type}{Description}
  \lineiii{\member{svn_module}}{\member{string}}{When using subversion as fetch method you can use this variable to specify a certain path in the repository. If this variable is not set skolesys-deb will try to use the module_name as path.}
  \lineiii{\member{svn_repos}}{\member{string}}{Specify the subversion repository to fetch the package source from. If this variable is set it will override the SKOLESYS_SVNBASE environment variable but not the command line argument -r.}
  \lineiii{\member{prebuild_script}}{\member{string}}{Script to be executed after content extraction from svn, iso or tgz and before the deb-file creation (see \member{skolesys-qt3_dapper_debinfo.py}) }
  \lineiii{\member{links}}{\member{dict}}{Each key-value entry in the links dict specifies a symbolic link to be created in the filesystem after installation.}
  \lineiii{\member{perm}}{\member{dict}}{The perm variable can be used to setup file permissions for the files being installed.}
  \lineiii{\member{preinst}}{\member{string}}{Script to be executed before package installation.}
  \lineiii{\member{postinst}}{\member{string}}{Script to be executed after package installation.}
  \lineiii{\member{prerm}}{\member{string}}{Script to be executed before package removal.}
  \lineiii{\member{postrm}}{\member{string}}{Script to be executed after package removal.}
\end{tableiii}
\citetitle{Optional Variables}

\index{debinfo example}
Example:
\begin{verbatim}
fetch_method = "svn"
svn_module = "system"

control = {
	'Package': 'python-skolesys-client',
	'Version': 'file://skolesys_ver',
	'NameExtension': 'feisty_all',
	'Section': 'python',
	'Priority': 'optional',
	'Architecture': 'all',
	'Depends': 'python-support (>= 0.2), python-soappy, python-m2crypto',
	'Recommends': 'skolesys-qt4',
	'Maintainer': 'Jakob Simon-Gaarde <jakob@skolesys.dk>',
	'Replaces': 'python-skolesys-seeder, python2.4-skolesys-seeder, python2.4-skolesys-client',
	'Conflicts': 'python2.4-skolesys-mainserver, python2.4-skolesys-seeder, python-skolesys-mainserver, python-skolesys-seeder',
	'Provides': 'python2.5-skolesys-client, python2.3-skolesys-client, python2.4-skolesys-client',
	'Description': 'This is the soap client part of the SkoleSYS linux distribution',
	'longdesc': 
""" The skolesys package provides the nessecary tools for administrating the SkoleSYS
 distribution. The main issue here is creating users and groups, controlling permissions,
 creating user and group spaces, registering client workstations (Windows, Linux, MacOS)
 and registering thin client servers (LTSP).
"""}

perm = {'cfmachine/cfinstaller.py': '755',
	'soap/getconf.py': '755',
	'soap/reghost.py': '755'}

copy = {
	'__init__.py': '/usr/share/python-support/python-skolesys-client/skolesys/',
	'soap/__init__.py': '/usr/share/python-support/python-skolesys-client/skolesys/soap',
	'soap/netinfo.py': '/usr/share/python-support/python-skolesys-client/skolesys/soap',
	'soap/marshall.py': '/usr/share/python-support/python-skolesys-client/skolesys/soap',
	'soap/getconf.py': '/usr/share/python-support/python-skolesys-client/skolesys/soap',
	'soap/reghost.py': '/usr/share/python-support/python-skolesys-client/skolesys/soap',
	'soap/client.py': '/usr/share/python-support/python-skolesys-client/skolesys/soap',
	'soap/p2.py': '/usr/share/python-support/python-skolesys-client/skolesys/soap',
	'cfmachine/__init__.py': '/usr/share/python-support/python-skolesys-client/skolesys/cfmachine',
	'cfmachine/cfinstaller.py': '/usr/share/python-support/python-skolesys-client/skolesys/cfmachine',
	'cfmachine/apthelpers.py': '/usr/share/python-support/python-skolesys-client/skolesys/cfmachine',
	'cfmachine/fstabhelpers.py': '/usr/share/python-support/python-skolesys-client/skolesys/cfmachine',
	'tools': '/usr/share/python-support/python-skolesys-client/skolesys/',
	'definitions': '/usr/share/python-support/python-skolesys-client/skolesys/'}

links = {
	'/usr/sbin/ss_installer': '../share/python-support/python-skolesys-client/skolesys/cfmachine/cfinstaller.py',
	'/usr/sbin/ss_getconf': '../share/python-support/python-skolesys-client/skolesys/soap/getconf.py',
	'/usr/sbin/ss_reghost': '../share/python-support/python-skolesys-client/skolesys/soap/reghost.py'}

postinst = """#!/bin/sh
set -e
# Automatically added by dh_pysupport
if [ "$1" = "configure" ] && which update-python-modules >/dev/null 2>&1; then
        update-python-modules -i /usr/share/python-support/python-skolesys-client
fi
# End automatically added section
"""

prerm = """#!/bin/sh
set -e
# Automatically added by dh_pysupport
if which update-python-modules >/dev/null 2>&1; then
        update-python-modules -c -i /usr/share/python-support/python-skolesys-client
fi
# End automatically added section
"""

postrm = """#!/bin/sh
if [ -e /usr/share/python-support/python-skolesys-client/skolesys ]
then
  find /usr/share/python-support/python-skolesys-client/skolesys -name "*.pyc" -delete
  find /usr/share/python-support/python-skolesys-client/skolesys -name "*.pyo" -delete
fi
"""
\end{verbatim}

\subsection{Subversion based modules\label{skolesys-deb-svn}}
Most of the skolesys-deb modules rely on fetching their content from a subversion repository. It is possible to setup this repository once and for all by setting the SKOLESYS_SVNBASE environment variable. 

However, if you need to fetch content for different skolesys-deb modules from different repositories you can specify the repository for each module by setting the svn_repos variable in the debinfo file or you can do it by using the -r command line option. svn_repos will override SKOLESYS_SVNBASE, -r command line options overrides everything.
The debinfo variable svn_module specifies the path to fetch from inside the repository. If svn_module is unset skolesys-deb will try to fetch using the module_name as path.

\begin{tableii}{c|l}{}{Environment Variable}{Description}
  \lineii{\member{SKOLESYS_SVNBASE}}{The default subversion repository or repository parent dir.}
\end{tableii}

\note{If you need to fetch from the base of a repository you will have to set svn_repos to the parent dir of the repository - so if I need to fetch from the root of the repository located at svn.mydomain.org/srv/svn/skolesys I will need to set svn_repos to somthing like svn+ssh://svn.mydomain.org/srv/svn/ and svn_module to skolesys.}

\subsection{ISO and tar.gz based modules\label{skolesys-deb-iso}}
There is not much to say about these fetch methods. All skolesys-deb needs to know is where to find the iso or tgz files. Like when using subversion for fetch method you can set environment variables that will serve as default directories for iso/tgz file hunting.

command line options -l iso_dir and -t  tgz_dir can also be used to set the directory location and these will of course override the environment variables if set. 

\begin{tableii}{c|l}{}{Environment Variable}{Description}
  \lineii{\member{SKOLESYS_ISODIR}}{The default directory when searching for ISO files.}
  \lineii{\member{SKOLESYS_TGZDIR}}{The default directory when searching for tgz files.}
\end{tableii}

\section{skolesys-apt\label{skolesys-apt}}
After creating debian packages the next thing needed is to make life easier for debian users who want to use the packages. To do that we are going to create a standard debian package repository which debian APT (Advanced Package Tool) relies on. 

skolesys-apt can do this automatically, all we need to do is create a simple aptinfo file and place the debian files in a distribution hierarchi - also we need a gpg key-pair to sign the APT Release file. If you don't have such a key-pair it will be shown how to gererate it.

skolesys-apt will create the APT repository on the local filesystem ready to ship by copying the two folders \member{dists} and \member{pool} and the public part of the gpg key-pair to the host running the webserver that is going to expose the repository.

\index{skolesys-apt}
\member{Usage: skolesys-apt dist_name}

\subsection{Prepair gpg key-pair\label{skolesys-apt-gpg}}

Prerequisite: You need \member{gnupg} installed to go any further.

\subsubsection{Reuse an existing gpg key-pair\label{skolesys-apt-reusegpg}}
If you already have a gpg key-pair you wish to use for signing the APT repository it os possible to export the key-pair and then import it on the host where you are running skolesys-apt.
Let us say that the user gpg-owner on host gpg-source-host has the key-pair we want to export and the user skolesys-devel on host gpg-dest-host needs to import it:

Export key-pair:
\begin{enumerate}
 \item log on to the source host gpg-owner@gpg-source-host
 \item Export the gpg key-pair:
\begin{verbatim}
gpg-owner@gpg-source-host$ gpg --armor --export-secret-keys EEF2B7FA > secret.gpg.asc
gpg-owner@gpg-source-host$ gpg --armor --export EEF2B7FA > public.gpg.asc
\end{verbatim}
\end{enumerate}
\note{EEF2B7FA is the key ID. Use gpg --list-keys to view your key ID's}

Import key-pair:

\begin{enumerate}
 \item Copy the secret.gpg.asc and public.gpg.asc from the source host
 \item log on to the destination host skolesys-developer@gpg-dest-host
 \item Import the key-pair
\begin{verbatim}
skolesys-devel@gpg-dest-host$ gpg --allow-secret-key-import --import public.gpg.asc secret.gpg.asc
\end{verbatim}
\end{enumerate}

\member{public.gpg.asc} will be the public key block that you must expose to the public to make your signed APT repository validate and thereby accessible to remote debian based systems.

\subsubsection{Create a gpg key-pair\label{skolesys-apt-creategpg}}
If you don't have a gpg key-pair to use for this task then you need to create one now.
\begin{enumerate}
 \item Create the gpg key-pair
 \begin{verbatim}
gpg --gen-key

.... (I used the no-existing email: jakob@email.com)

pub   1024D/55117311 2007-06-21
	Key fingerprint = F461 E5AF 0DE8 B3E4 358D  DEB6 EC23 7CF4 5511 7311
uid                  Jakob Simon-Gaarde <jakob@email.com>
sub   2048g/FDD61F6E 2007-06-21
 \end{verbatim}
 \note{It is not required that you supply a passphrase you can leave it empty.}

 \item In the example above the newly generated key-pair has ID 55117311
 \item Extract a public key block	
 \begin{verbatim}
gpg --export --armor jakob@email.com > public.gpg.asc
 \end{verbatim}

 \member{public.gpg.asc} will be the public key block that you must expose to the public to make your signed APT repository validate and thereby accessible to remote debian based systems.

\end{enumerate}

\subsection{Creating a distribution hierarchi\label{skolesys-apt-pool}}
Most of the skolesys-deb modules rely on fetching their content from a subversion repository. It is possible to setup this repository once and for all by setting the SKOLESYS_SVNBASE environment variable. 

However, if you need to fetch content for different skolesys-deb modules from different repositories you can specify the repository for each module by setting the svn_repos variable in the debinfo file or you can do it by using the -r command line option. svn_repos will override SKOLESYS_SVNBASE, -r command line options overrides everything.
The debinfo variable svn_module specifies the path to fetch from inside the repository. If svn_module is unset skolesys-deb will try to fetch using the module_name as path.

\begin{tableii}{c|l}{}{Environment Variable}{Description}
  \lineii{\member{SKOLESYS_SVNBASE}}{The default subversion repository or repository parent dir.}
\end{tableii}

\note{If you need to fetch from the base of a repository you will have to set svn_repos to the parent dir of the repository - so if I need to fetch from the root of the repository located at svn.mydomain.org/srv/svn/skolesys I will need to set svn_repos to somthing like svn+ssh://svn.mydomain.org/srv/svn/ and svn_module to skolesys.}

\subsection{Creating aptinfo file\label{skolesys-apt-aptinfo}}
There is not much to say about these fetch methods. All skolesys-deb needs to know is where to find the iso or tgz files. Like when using subversion for fetch method you can set environment variables that will serve as default directories for iso/tgz file hunting.

command line options -l iso_dir and -t  tgz_dir can also be used to set the directory location and these will of course override the environment variables if set. 

\begin{tableii}{c|l}{}{Environment Variable}{Description}
  \lineii{\member{SKOLESYS_ISODIR}}{The default directory when searching for ISO files.}
  \lineii{\member{SKOLESYS_TGZDIR}}{The default directory when searching for tgz files.}
\end{tableii}



\documentclass{manual}

\title{SkoleSYS Developer Documentation}

\makeindex			% tell \index to actually write the .idx file


\begin{document}

\maketitle

\ifhtml
\chapter*{Front Matter\label{front}}
\fi

\begin{abstract}

\noindent
This document addresses the developer specific issues of SkoleSYS. It is
split up in two main parts - SkoleSYS library documentation and SkoleSYS
distribution tools (disttools).

SkoleSYS Library Documentation coveres the skolesys python package wich
is essentially all SkoleSYS logic e.g. client, server, configuration, SOAP
and more.

SkoleSYS Distribution Tools documents the three tools for deb-packaging,
APT distribution and install CD creation (skolesys-deb, skolesys-apt and 
skolesys-cd)

\end{abstract}

\tableofcontents

\chapter{SkoleSYS Distribution Tools (trunk/disttools)\label{disttools}}
The SkoleSYS Distribution Tools are just convenience tools I have made to make the job of distributing SkoleSYS fast and easy. That way I can keep the main focus on developing the SkoleSYS libraries and GUI. 

The \member{skolesys-apt} script can fetch content via subversion, tarballs or iso-files and place data where you want it redistributed on target filesystems. You can define symlinks and install-scripts (preinst, postinst, prerm, postrm) and setup file permissions. It is also possible to register a scripts to be run by \member{skolesys-apt} after content extraction (ie. pre-bytecompile python scripts). Once the debinfo files (control files for \member{skolesys-apt}) are in place building a file is as easy as executing one command - you can even setup automatic version ticking and group debian packages together by syncronizing their version number.

After building debian packages \member{skolesys-apt} is the tool for building an APT archive. You only need to setup one aptinfo file (control file for \member{skolesys-apt}) per distribution you target create a \member{distribution hieararchi}\footnote{A distribution hieararchi tells \member{skolesys-apt} which packages go into which distribution components and which computer architecture they are build for.}. \member{skolesys-apt} will automatically generate the \member{dists} and \member{pool} directory so they are ready to put on your distribution webserver.
\section{skolesys-deb\label{skolesys-deb}}
This tool is used to create deb-files for distribution. It can fetch distribution content using svn, iso or tgz. Each deb-file is controlled by a debinfo file, the naming convention for such a file is \member{<module_name>_debinfo.py} - e.g. the module named \member{"python-skolesys-client_feisty"} should be named: \member{"python-skolesys-client_feisty_debinfo.py"}. Module name should not be confused with package name, the package name is specified inside the debinfo file.

\index{skolesys-deb}
\member{Usage: skolesys-deb [options] module_name}


\begin{tableiii}{l|l|l}{command}{Short-option}{Long-option}{Description}
  \lineiii{-r \var{url}}{\member{--svnbase=\var{url}}}{URL to the subversion repository to fetch from (use when \member{fetch_method='svn'})}
  \lineiii{\member{-l \var{dir}}}{\member{--iso-location=\var{dir}}}{The directory containing the iso file (use when \member{fetch_method='iso'})}
  \lineiii{\member{-l \var{dir}}}{\member{--iso-location=\var{dir}}}{The directory containing the tar.gz file (use when \member{fetch_method='tgz'})}
  \lineiii{}{\member{--dont-fetch}}{Don't fetch the package contents. This is used if more than one package is based on the same content resource, so if these packages are build successively there is no need to fetch the content more than once.}
  \lineiii{\member{-h}}{\member{--help}}{Show the help message and exit}
\end{tableiii}
\citetitle{skolesys-deb command line options}
\subsection{debinfo files\label{debinfo}}
Describing the naming convention has already revealed that the python interpreter is used for parsing debinfo files, and that is also why I say module name instead of package name. A debinfo file consists of a number of global variables, most of these are optional but there are three mandatory:

\index{debinfo file options}
\begin{tableiii}{c|l|l}{textrm}{Name}{Type}{Description}
  \lineiii{\member{fetch_method}}{\member{string}}{Specify how to fetch the content for the deb-package.}
  \lineiii{\member{control}}{\member{dict}}{This dictionary holds the normal dpkg control structure. There is a special feature regarding the Version value for grouping packages together giving them the same package version. Simply set the value to point to the file containing the version number in the first line. (e.g. 'Version': 'file://ver_no')}
  \lineiii{\member{copy}}{\member{dict}}{Specify which files should go where in the filesystem during installation.}
\end{tableiii}
\citetitle{Mandatory variables}

\begin{tableiii}{c|l|l}{textrm}{Name}{Type}{Description}
  \lineiii{\member{svn_module}}{\member{string}}{When using subversion as fetch method you can use this variable to specify a certain path in the repository. If this variable is not set skolesys-deb will try to use the module_name as path.}
  \lineiii{\member{svn_repos}}{\member{string}}{Specify the subversion repository to fetch the package source from. If this variable is set it will override the SKOLESYS_SVNBASE environment variable but not the command line argument -r.}
  \lineiii{\member{prebuild_script}}{\member{string}}{Script to be executed after content extraction from svn, iso or tgz and before the deb-file creation (see \member{skolesys-qt3_dapper_debinfo.py}) }
  \lineiii{\member{links}}{\member{dict}}{Each key-value entry in the links dict specifies a symbolic link to be created in the filesystem after installation.}
  \lineiii{\member{perm}}{\member{dict}}{The perm variable can be used to setup file permissions for the files being installed.}
  \lineiii{\member{preinst}}{\member{string}}{Script to be executed before package installation.}
  \lineiii{\member{postinst}}{\member{string}}{Script to be executed after package installation.}
  \lineiii{\member{prerm}}{\member{string}}{Script to be executed before package removal.}
  \lineiii{\member{postrm}}{\member{string}}{Script to be executed after package removal.}
\end{tableiii}
\citetitle{Optional Variables}

\index{debinfo example}
Example:
\begin{verbatim}
fetch_method = "svn"
svn_module = "system"

control = {
	'Package': 'python-skolesys-client',
	'Version': 'file://skolesys_ver',
	'NameExtension': 'feisty_all',
	'Section': 'python',
	'Priority': 'optional',
	'Architecture': 'all',
	'Depends': 'python-support (>= 0.2), python-soappy, python-m2crypto',
	'Recommends': 'skolesys-qt4',
	'Maintainer': 'Jakob Simon-Gaarde <jakob@skolesys.dk>',
	'Replaces': 'python-skolesys-seeder, python2.4-skolesys-seeder, python2.4-skolesys-client',
	'Conflicts': 'python2.4-skolesys-mainserver, python2.4-skolesys-seeder, python-skolesys-mainserver, python-skolesys-seeder',
	'Provides': 'python2.5-skolesys-client, python2.3-skolesys-client, python2.4-skolesys-client',
	'Description': 'This is the soap client part of the SkoleSYS linux distribution',
	'longdesc': 
""" The skolesys package provides the nessecary tools for administrating the SkoleSYS
 distribution. The main issue here is creating users and groups, controlling permissions,
 creating user and group spaces, registering client workstations (Windows, Linux, MacOS)
 and registering thin client servers (LTSP).
"""}

perm = {'cfmachine/cfinstaller.py': '755',
	'soap/getconf.py': '755',
	'soap/reghost.py': '755'}

copy = {
	'__init__.py': '/usr/share/python-support/python-skolesys-client/skolesys/',
	'soap/__init__.py': '/usr/share/python-support/python-skolesys-client/skolesys/soap',
	'soap/netinfo.py': '/usr/share/python-support/python-skolesys-client/skolesys/soap',
	'soap/marshall.py': '/usr/share/python-support/python-skolesys-client/skolesys/soap',
	'soap/getconf.py': '/usr/share/python-support/python-skolesys-client/skolesys/soap',
	'soap/reghost.py': '/usr/share/python-support/python-skolesys-client/skolesys/soap',
	'soap/client.py': '/usr/share/python-support/python-skolesys-client/skolesys/soap',
	'soap/p2.py': '/usr/share/python-support/python-skolesys-client/skolesys/soap',
	'cfmachine/__init__.py': '/usr/share/python-support/python-skolesys-client/skolesys/cfmachine',
	'cfmachine/cfinstaller.py': '/usr/share/python-support/python-skolesys-client/skolesys/cfmachine',
	'cfmachine/apthelpers.py': '/usr/share/python-support/python-skolesys-client/skolesys/cfmachine',
	'cfmachine/fstabhelpers.py': '/usr/share/python-support/python-skolesys-client/skolesys/cfmachine',
	'tools': '/usr/share/python-support/python-skolesys-client/skolesys/',
	'definitions': '/usr/share/python-support/python-skolesys-client/skolesys/'}

links = {
	'/usr/sbin/ss_installer': '../share/python-support/python-skolesys-client/skolesys/cfmachine/cfinstaller.py',
	'/usr/sbin/ss_getconf': '../share/python-support/python-skolesys-client/skolesys/soap/getconf.py',
	'/usr/sbin/ss_reghost': '../share/python-support/python-skolesys-client/skolesys/soap/reghost.py'}

postinst = """#!/bin/sh
set -e
# Automatically added by dh_pysupport
if [ "$1" = "configure" ] && which update-python-modules >/dev/null 2>&1; then
        update-python-modules -i /usr/share/python-support/python-skolesys-client
fi
# End automatically added section
"""

prerm = """#!/bin/sh
set -e
# Automatically added by dh_pysupport
if which update-python-modules >/dev/null 2>&1; then
        update-python-modules -c -i /usr/share/python-support/python-skolesys-client
fi
# End automatically added section
"""

postrm = """#!/bin/sh
if [ -e /usr/share/python-support/python-skolesys-client/skolesys ]
then
  find /usr/share/python-support/python-skolesys-client/skolesys -name "*.pyc" -delete
  find /usr/share/python-support/python-skolesys-client/skolesys -name "*.pyo" -delete
fi
"""
\end{verbatim}

\subsection{Subversion based modules\label{skolesys-deb-svn}}
Most of the skolesys-deb modules rely on fetching their content from a subversion repository. It is possible to setup this repository once and for all by setting the SKOLESYS_SVNBASE environment variable. 

However, if you need to fetch content for different skolesys-deb modules from different repositories you can specify the repository for each module by setting the svn_repos variable in the debinfo file or you can do it by using the -r command line option. svn_repos will override SKOLESYS_SVNBASE, -r command line options overrides everything.
The debinfo variable svn_module specifies the path to fetch from inside the repository. If svn_module is unset skolesys-deb will try to fetch using the module_name as path.

\begin{tableii}{c|l}{}{Environment Variable}{Description}
  \lineii{\member{SKOLESYS_SVNBASE}}{The default subversion repository or repository parent dir.}
\end{tableii}

\note{If you need to fetch from the base of a repository you will have to set svn_repos to the parent dir of the repository - so if I need to fetch from the root of the repository located at svn.mydomain.org/srv/svn/skolesys I will need to set svn_repos to somthing like svn+ssh://svn.mydomain.org/srv/svn/ and svn_module to skolesys.}

\subsection{ISO and tar.gz based modules\label{skolesys-deb-iso}}
There is not much to say about these fetch methods. All skolesys-deb needs to know is where to find the iso or tgz files. Like when using subversion for fetch method you can set environment variables that will serve as default directories for iso/tgz file hunting.

command line options -l iso_dir and -t  tgz_dir can also be used to set the directory location and these will of course override the environment variables if set. 

\begin{tableii}{c|l}{}{Environment Variable}{Description}
  \lineii{\member{SKOLESYS_ISODIR}}{The default directory when searching for ISO files.}
  \lineii{\member{SKOLESYS_TGZDIR}}{The default directory when searching for tgz files.}
\end{tableii}

\section{skolesys-apt\label{skolesys-apt}}
After creating debian packages the next thing needed is to make life easier for debian users who want to use the packages. To do that we are going to create a standard debian package repository which debian APT (Advanced Package Tool) relies on. 

skolesys-apt can do this automatically, all we need to do is create a simple aptinfo file and place the debian files in a distribution hierarchi - also we need a gpg key-pair to sign the APT Release file. If you don't have such a key-pair it will be shown how to gererate it.

skolesys-apt will create the APT repository on the local filesystem ready to ship by copying the two folders \member{dists} and \member{pool} and the public part of the gpg key-pair to the host running the webserver that is going to expose the repository.

\index{skolesys-apt}
\member{Usage: skolesys-apt dist_name}

\subsection{Prepair gpg key-pair\label{skolesys-apt-gpg}}

Prerequisite: You need \member{gnupg} installed to go any further.

\subsubsection{Reuse an existing gpg key-pair\label{skolesys-apt-reusegpg}}
If you already have a gpg key-pair you wish to use for signing the APT repository it os possible to export the key-pair and then import it on the host where you are running skolesys-apt.
Let us say that the user gpg-owner on host gpg-source-host has the key-pair we want to export and the user skolesys-devel on host gpg-dest-host needs to import it:

Export key-pair:
\begin{enumerate}
 \item log on to the source host gpg-owner@gpg-source-host
 \item Export the gpg key-pair:
\begin{verbatim}
gpg-owner@gpg-source-host$ gpg --armor --export-secret-keys EEF2B7FA > secret.gpg.asc
gpg-owner@gpg-source-host$ gpg --armor --export EEF2B7FA > public.gpg.asc
\end{verbatim}
\end{enumerate}
\note{EEF2B7FA is the key ID. Use gpg --list-keys to view your key ID's}

Import key-pair:

\begin{enumerate}
 \item Copy the secret.gpg.asc and public.gpg.asc from the source host
 \item log on to the destination host skolesys-developer@gpg-dest-host
 \item Import the key-pair
\begin{verbatim}
skolesys-devel@gpg-dest-host$ gpg --allow-secret-key-import --import public.gpg.asc secret.gpg.asc
\end{verbatim}
\end{enumerate}

\member{public.gpg.asc} will be the public key block that you must expose to the public to make your signed APT repository validate and thereby accessible to remote debian based systems.

\subsubsection{Create a gpg key-pair\label{skolesys-apt-creategpg}}
If you don't have a gpg key-pair to use for this task then you need to create one now.
\begin{enumerate}
 \item Create the gpg key-pair
 \begin{verbatim}
gpg --gen-key

.... (I used the no-existing email: jakob@email.com)

pub   1024D/55117311 2007-06-21
	Key fingerprint = F461 E5AF 0DE8 B3E4 358D  DEB6 EC23 7CF4 5511 7311
uid                  Jakob Simon-Gaarde <jakob@email.com>
sub   2048g/FDD61F6E 2007-06-21
 \end{verbatim}
 \note{It is not required that you supply a passphrase you can leave it empty.}

 \item In the example above the newly generated key-pair has ID 55117311
 \item Extract a public key block	
 \begin{verbatim}
gpg --export --armor jakob@email.com > public.gpg.asc
 \end{verbatim}

 \member{public.gpg.asc} will be the public key block that you must expose to the public to make your signed APT repository validate and thereby accessible to remote debian based systems.

\end{enumerate}

\subsection{Creating a distribution hierarchi\label{skolesys-apt-pool}}
Most of the skolesys-deb modules rely on fetching their content from a subversion repository. It is possible to setup this repository once and for all by setting the SKOLESYS_SVNBASE environment variable. 

However, if you need to fetch content for different skolesys-deb modules from different repositories you can specify the repository for each module by setting the svn_repos variable in the debinfo file or you can do it by using the -r command line option. svn_repos will override SKOLESYS_SVNBASE, -r command line options overrides everything.
The debinfo variable svn_module specifies the path to fetch from inside the repository. If svn_module is unset skolesys-deb will try to fetch using the module_name as path.

\begin{tableii}{c|l}{}{Environment Variable}{Description}
  \lineii{\member{SKOLESYS_SVNBASE}}{The default subversion repository or repository parent dir.}
\end{tableii}

\note{If you need to fetch from the base of a repository you will have to set svn_repos to the parent dir of the repository - so if I need to fetch from the root of the repository located at svn.mydomain.org/srv/svn/skolesys I will need to set svn_repos to somthing like svn+ssh://svn.mydomain.org/srv/svn/ and svn_module to skolesys.}

\subsection{Creating aptinfo file\label{skolesys-apt-aptinfo}}
There is not much to say about these fetch methods. All skolesys-deb needs to know is where to find the iso or tgz files. Like when using subversion for fetch method you can set environment variables that will serve as default directories for iso/tgz file hunting.

command line options -l iso_dir and -t  tgz_dir can also be used to set the directory location and these will of course override the environment variables if set. 

\begin{tableii}{c|l}{}{Environment Variable}{Description}
  \lineii{\member{SKOLESYS_ISODIR}}{The default directory when searching for ISO files.}
  \lineii{\member{SKOLESYS_TGZDIR}}{The default directory when searching for tgz files.}
\end{tableii}

%\section{skolesys-cd\label{skolesys-cd}}
This tool is used to create deb-files for distribution. It can fetch distribution content using svn, iso or tgz. Each deb-file is controlled by a debinfo file, the naming convention for such a file is \member{<module_name>_debinfo.py} - e.g. the module named \member{"python-skolesys-client_feisty"} should be named: \member{"python-skolesys-client_feisty_debinfo.py"}. Module name should not be confused with package name, the package name is specified inside the debinfo file.

\begin{verbatim}
skolesys-deb[-r svn_repos][-l iso_dir][-t tgz_dir][--dont-fetch] module_name
\end{verbatim}


\begin{tableiii}{l|l|l}{command}{Short-option}{Long-option}{Description}
  \lineiii{-r \var{url}}{\member{--svnbase=\var{url}}}{URL to the subversion repository to fetch from (use when \member{fetch_method='svn'})}
  \lineiii{\member{-l \var{dir}}}{\member{--iso-location=\var{dir}}}{The directory containing the iso file (use when \member{fetch_method='iso'})}
  \lineiii{\member{-l \var{dir}}}{\member{--iso-location=\var{dir}}}{The directory containing the tar.gz file (use when \member{fetch_method='tgz'})}
  \lineiii{}{\member{--dont-fetch}}{Don't fetch the package contents. This is used if more than one package is based on the same content resource, so if these packages are build successively there is no need to fetch the content more than once.}
  \lineiii{\member{-h}}{\member{--help}}{Show the help message and exit}
\end{tableiii}

\subsection{debinfo files\label{debinfo}}
Describing the naming convention has already revealed that the python interpreter is used for parsing debinfo files, and that is also why I say module name instead of package name. A debinfo file consists of a number of global variables, most of these are optional but there are three mandatory.

Mandatory variables:
\begin{tableiii}{c|l|l}{textrm}{Name}{Type}{Description}
  \lineiii{\member{fetch_method}}{\member{string}}{Specify how to fetch the content for the deb-package.}
  \lineiii{\member{control}}{\member{dict}}{This dictionary holds the normal dpkg control structure. There is a special feature regarding the Version value for grouping packages together giving them the same package version. Simply set the value to point to the file containing the version number in the first line. (e.g. 'Version': 'file://ver_no')}
  \lineiii{\member{copy}}{\member{dict}}{Specify which files should go where in the filesystem during installation.}
\end{tableiii}

Optional Variables:
\begin{tableiii}{c|l|l}{textrm}{Name}{Type}{Description}
  \lineiii{\member{svn_module}}{\member{string}}{When using subversion as fetch method you can use this variable to specify a certain path in the repository. If this variable is not set skolesys-deb will try to use the module_name as path.}
  \lineiii{\member{svn_repos}}{\member{string}}{Specify the subversion repository to fetch the package source from. If this variable is set it will override the SKOLESYS_SVNBASE environment variable but not the command line argument -r.}
  \lineiii{\member{links}}{\member{dict}}{Each key-value entry in the links dict specifies a symbolic link to be created in the filesystem after installation.}
  \lineiii{\member{perm}}{\member{dict}}{The perm variable can be used to setup file permissions for the files being installed.}
  \lineiii{\member{preinst}}{\member{string}}{Script to be executed before package installation.}
  \lineiii{\member{postinst}}{\member{string}}{Script to be executed after package installation.}
  \lineiii{\member{prerm}}{\member{string}}{Script to be executed before package removal.}
  \lineiii{\member{postrm}}{\member{string}}{Script to be executed after package removal.}
\end{tableiii}

Example:
\begin{verbatim}
fetch_method = "svn"
svn_module = "system"

control = {
	'Package': 'python-skolesys-client',
	'Version': 'file://skolesys_ver',
	'NameExtension': 'feisty_all',
	'Section': 'python',
	'Priority': 'optional',
	'Architecture': 'all',
	'Depends': 'python-support (>= 0.2), python-soappy, python-m2crypto',
	'Recommends': 'skolesys-qt4',
	'Maintainer': 'Jakob Simon-Gaarde <jakob@skolesys.dk>',
	'Replaces': 'python-skolesys-seeder, python2.4-skolesys-seeder, python2.4-skolesys-client',
	'Conflicts': 'python2.4-skolesys-mainserver, python2.4-skolesys-seeder, python-skolesys-mainserver, python-skolesys-seeder',
	'Provides': 'python2.5-skolesys-client, python2.3-skolesys-client, python2.4-skolesys-client',
	'Description': 'This is the soap client part of the SkoleSYS linux distribution',
	'longdesc': 
""" The skolesys package provides the nessecary tools for administrating the SkoleSYS
 distribution. The main issue here is creating users and groups, controlling permissions,
 creating user and group spaces, registering client workstations (Windows, Linux, MacOS)
 and registering thin client servers (LTSP).
"""}

perm = {'cfmachine/cfinstaller.py': '755',
	'soap/getconf.py': '755',
	'soap/reghost.py': '755'}

copy = {
	'__init__.py': '/usr/share/python-support/python-skolesys-client/skolesys/',
	'soap/__init__.py': '/usr/share/python-support/python-skolesys-client/skolesys/soap',
	'soap/netinfo.py': '/usr/share/python-support/python-skolesys-client/skolesys/soap',
	'soap/marshall.py': '/usr/share/python-support/python-skolesys-client/skolesys/soap',
	'soap/getconf.py': '/usr/share/python-support/python-skolesys-client/skolesys/soap',
	'soap/reghost.py': '/usr/share/python-support/python-skolesys-client/skolesys/soap',
	'soap/client.py': '/usr/share/python-support/python-skolesys-client/skolesys/soap',
	'soap/p2.py': '/usr/share/python-support/python-skolesys-client/skolesys/soap',
	'cfmachine/__init__.py': '/usr/share/python-support/python-skolesys-client/skolesys/cfmachine',
	'cfmachine/cfinstaller.py': '/usr/share/python-support/python-skolesys-client/skolesys/cfmachine',
	'cfmachine/apthelpers.py': '/usr/share/python-support/python-skolesys-client/skolesys/cfmachine',
	'cfmachine/fstabhelpers.py': '/usr/share/python-support/python-skolesys-client/skolesys/cfmachine',
	'tools': '/usr/share/python-support/python-skolesys-client/skolesys/',
	'definitions': '/usr/share/python-support/python-skolesys-client/skolesys/'}

links = {
	'/usr/sbin/ss_installer': '../share/python-support/python-skolesys-client/skolesys/cfmachine/cfinstaller.py',
	'/usr/sbin/ss_getconf': '../share/python-support/python-skolesys-client/skolesys/soap/getconf.py',
	'/usr/sbin/ss_reghost': '../share/python-support/python-skolesys-client/skolesys/soap/reghost.py'}

postinst = """#!/bin/sh
set -e
# Automatically added by dh_pysupport
if [ "$1" = "configure" ] && which update-python-modules >/dev/null 2>&1; then
        update-python-modules -i /usr/share/python-support/python-skolesys-client
fi
# End automatically added section
"""

prerm = """#!/bin/sh
set -e
# Automatically added by dh_pysupport
if which update-python-modules >/dev/null 2>&1; then
        update-python-modules -c -i /usr/share/python-support/python-skolesys-client
fi
# End automatically added section
"""

postrm = """#!/bin/sh
if [ -e /usr/share/python-support/python-skolesys-client/skolesys ]
then
  find /usr/share/python-support/python-skolesys-client/skolesys -name "*.pyc" -delete
  find /usr/share/python-support/python-skolesys-client/skolesys -name "*.pyo" -delete
fi
"""
\end{verbatim}

\subsection{Subversion based modules\label{skolesys-deb-svn}}
Most of the skolesys-deb modules rely on fetching their content from a subversion repository. It is possible to setup this repository once and for all by setting the SKOLESYS_SVNBASE environment variable. 

However, if you need to fetch content for different skolesys-deb modules from different repositories you can specify the repository for each module by setting the svn_repos variable in the debinfo file or you can do it by using the -r command line option. svn_repos will override SKOLESYS_SVNBASE, -r command line options overrides everything.
The debinfo variable svn_module specifies the path to fetch from inside the repository. If svn_module is unset skolesys-deb will try to fetch using the module_name as path.

\begin{tableii}{c|l}{}{Environment Variable}{Description}
  \lineii{\member{SKOLESYS_SVNBASE}}{The default subversion repository or repository parent dir.}
\end{tableii}

\note{If you need to fetch from the base of a repository you will have to set svn_repos to the parent dir of the repository - so if I need to fetch from the root of the repository located at svn.mydomain.org/srv/svn/skolesys I will need to set svn_repos to somthing like svn+ssh://svn.mydomain.org/srv/svn/ and svn_module to skolesys.}

\subsection{ISO and tar.gz based modules\label{skolesys-deb-iso}}
There is not much to say about these fetch methods. All skolesys-deb needs to know is where to find the iso or tgz files. Like when using subversion for fetch method you can set environment variables that will serve as default directories for iso/tgz file hunting.

command line options -l iso_dir and -t  tgz_dir can also be used to set the directory location and these will of course override the environment variables if set. 

\begin{tableii}{c|l}{}{Environment Variable}{Description}
  \lineii{\member{SKOLESYS_ISODIR}}{The default directory when searching for ISO files.}
  \lineii{\member{SKOLESYS_TGZDIR}}{The default directory when searching for tgz files.}
\end{tableii}



% \chapter{Debugging \label{debugging}}
%
% XXX Explain Py_DEBUG, Py_TRACE_REFS, Py_REF_DEBUG.

\chapter{SkoleSYS Package (trunk/system)\label{skolesys-pack}}
\section{skolesys-deb\label{skolesys-deb}}
This tool is used to create deb-files for distribution. It can fetch distribution content using svn, iso or tgz. Each deb-file is controlled by a debinfo file, the naming convention for such a file is \member{<module_name>_debinfo.py} - e.g. the module named \member{"python-skolesys-client_feisty"} should be named: \member{"python-skolesys-client_feisty_debinfo.py"}. Module name should not be confused with package name, the package name is specified inside the debinfo file.

\index{skolesys-deb}
\member{Usage: skolesys-deb [options] module_name}


\begin{tableiii}{l|l|l}{command}{Short-option}{Long-option}{Description}
  \lineiii{-r \var{url}}{\member{--svnbase=\var{url}}}{URL to the subversion repository to fetch from (use when \member{fetch_method='svn'})}
  \lineiii{\member{-l \var{dir}}}{\member{--iso-location=\var{dir}}}{The directory containing the iso file (use when \member{fetch_method='iso'})}
  \lineiii{\member{-l \var{dir}}}{\member{--iso-location=\var{dir}}}{The directory containing the tar.gz file (use when \member{fetch_method='tgz'})}
  \lineiii{}{\member{--dont-fetch}}{Don't fetch the package contents. This is used if more than one package is based on the same content resource, so if these packages are build successively there is no need to fetch the content more than once.}
  \lineiii{\member{-h}}{\member{--help}}{Show the help message and exit}
\end{tableiii}
\citetitle{skolesys-deb command line options}
\subsection{debinfo files\label{debinfo}}
Describing the naming convention has already revealed that the python interpreter is used for parsing debinfo files, and that is also why I say module name instead of package name. A debinfo file consists of a number of global variables, most of these are optional but there are three mandatory:

\index{debinfo file options}
\begin{tableiii}{c|l|l}{textrm}{Name}{Type}{Description}
  \lineiii{\member{fetch_method}}{\member{string}}{Specify how to fetch the content for the deb-package.}
  \lineiii{\member{control}}{\member{dict}}{This dictionary holds the normal dpkg control structure. There is a special feature regarding the Version value for grouping packages together giving them the same package version. Simply set the value to point to the file containing the version number in the first line. (e.g. 'Version': 'file://ver_no')}
  \lineiii{\member{copy}}{\member{dict}}{Specify which files should go where in the filesystem during installation.}
\end{tableiii}
\citetitle{Mandatory variables}

\begin{tableiii}{c|l|l}{textrm}{Name}{Type}{Description}
  \lineiii{\member{svn_module}}{\member{string}}{When using subversion as fetch method you can use this variable to specify a certain path in the repository. If this variable is not set skolesys-deb will try to use the module_name as path.}
  \lineiii{\member{svn_repos}}{\member{string}}{Specify the subversion repository to fetch the package source from. If this variable is set it will override the SKOLESYS_SVNBASE environment variable but not the command line argument -r.}
  \lineiii{\member{prebuild_script}}{\member{string}}{Script to be executed after content extraction from svn, iso or tgz and before the deb-file creation (see \member{skolesys-qt3_dapper_debinfo.py}) }
  \lineiii{\member{links}}{\member{dict}}{Each key-value entry in the links dict specifies a symbolic link to be created in the filesystem after installation.}
  \lineiii{\member{perm}}{\member{dict}}{The perm variable can be used to setup file permissions for the files being installed.}
  \lineiii{\member{preinst}}{\member{string}}{Script to be executed before package installation.}
  \lineiii{\member{postinst}}{\member{string}}{Script to be executed after package installation.}
  \lineiii{\member{prerm}}{\member{string}}{Script to be executed before package removal.}
  \lineiii{\member{postrm}}{\member{string}}{Script to be executed after package removal.}
\end{tableiii}
\citetitle{Optional Variables}

\index{debinfo example}
Example:
\begin{verbatim}
fetch_method = "svn"
svn_module = "system"

control = {
	'Package': 'python-skolesys-client',
	'Version': 'file://skolesys_ver',
	'NameExtension': 'feisty_all',
	'Section': 'python',
	'Priority': 'optional',
	'Architecture': 'all',
	'Depends': 'python-support (>= 0.2), python-soappy, python-m2crypto',
	'Recommends': 'skolesys-qt4',
	'Maintainer': 'Jakob Simon-Gaarde <jakob@skolesys.dk>',
	'Replaces': 'python-skolesys-seeder, python2.4-skolesys-seeder, python2.4-skolesys-client',
	'Conflicts': 'python2.4-skolesys-mainserver, python2.4-skolesys-seeder, python-skolesys-mainserver, python-skolesys-seeder',
	'Provides': 'python2.5-skolesys-client, python2.3-skolesys-client, python2.4-skolesys-client',
	'Description': 'This is the soap client part of the SkoleSYS linux distribution',
	'longdesc': 
""" The skolesys package provides the nessecary tools for administrating the SkoleSYS
 distribution. The main issue here is creating users and groups, controlling permissions,
 creating user and group spaces, registering client workstations (Windows, Linux, MacOS)
 and registering thin client servers (LTSP).
"""}

perm = {'cfmachine/cfinstaller.py': '755',
	'soap/getconf.py': '755',
	'soap/reghost.py': '755'}

copy = {
	'__init__.py': '/usr/share/python-support/python-skolesys-client/skolesys/',
	'soap/__init__.py': '/usr/share/python-support/python-skolesys-client/skolesys/soap',
	'soap/netinfo.py': '/usr/share/python-support/python-skolesys-client/skolesys/soap',
	'soap/marshall.py': '/usr/share/python-support/python-skolesys-client/skolesys/soap',
	'soap/getconf.py': '/usr/share/python-support/python-skolesys-client/skolesys/soap',
	'soap/reghost.py': '/usr/share/python-support/python-skolesys-client/skolesys/soap',
	'soap/client.py': '/usr/share/python-support/python-skolesys-client/skolesys/soap',
	'soap/p2.py': '/usr/share/python-support/python-skolesys-client/skolesys/soap',
	'cfmachine/__init__.py': '/usr/share/python-support/python-skolesys-client/skolesys/cfmachine',
	'cfmachine/cfinstaller.py': '/usr/share/python-support/python-skolesys-client/skolesys/cfmachine',
	'cfmachine/apthelpers.py': '/usr/share/python-support/python-skolesys-client/skolesys/cfmachine',
	'cfmachine/fstabhelpers.py': '/usr/share/python-support/python-skolesys-client/skolesys/cfmachine',
	'tools': '/usr/share/python-support/python-skolesys-client/skolesys/',
	'definitions': '/usr/share/python-support/python-skolesys-client/skolesys/'}

links = {
	'/usr/sbin/ss_installer': '../share/python-support/python-skolesys-client/skolesys/cfmachine/cfinstaller.py',
	'/usr/sbin/ss_getconf': '../share/python-support/python-skolesys-client/skolesys/soap/getconf.py',
	'/usr/sbin/ss_reghost': '../share/python-support/python-skolesys-client/skolesys/soap/reghost.py'}

postinst = """#!/bin/sh
set -e
# Automatically added by dh_pysupport
if [ "$1" = "configure" ] && which update-python-modules >/dev/null 2>&1; then
        update-python-modules -i /usr/share/python-support/python-skolesys-client
fi
# End automatically added section
"""

prerm = """#!/bin/sh
set -e
# Automatically added by dh_pysupport
if which update-python-modules >/dev/null 2>&1; then
        update-python-modules -c -i /usr/share/python-support/python-skolesys-client
fi
# End automatically added section
"""

postrm = """#!/bin/sh
if [ -e /usr/share/python-support/python-skolesys-client/skolesys ]
then
  find /usr/share/python-support/python-skolesys-client/skolesys -name "*.pyc" -delete
  find /usr/share/python-support/python-skolesys-client/skolesys -name "*.pyo" -delete
fi
"""
\end{verbatim}

\subsection{Subversion based modules\label{skolesys-deb-svn}}
Most of the skolesys-deb modules rely on fetching their content from a subversion repository. It is possible to setup this repository once and for all by setting the SKOLESYS_SVNBASE environment variable. 

However, if you need to fetch content for different skolesys-deb modules from different repositories you can specify the repository for each module by setting the svn_repos variable in the debinfo file or you can do it by using the -r command line option. svn_repos will override SKOLESYS_SVNBASE, -r command line options overrides everything.
The debinfo variable svn_module specifies the path to fetch from inside the repository. If svn_module is unset skolesys-deb will try to fetch using the module_name as path.

\begin{tableii}{c|l}{}{Environment Variable}{Description}
  \lineii{\member{SKOLESYS_SVNBASE}}{The default subversion repository or repository parent dir.}
\end{tableii}

\note{If you need to fetch from the base of a repository you will have to set svn_repos to the parent dir of the repository - so if I need to fetch from the root of the repository located at svn.mydomain.org/srv/svn/skolesys I will need to set svn_repos to somthing like svn+ssh://svn.mydomain.org/srv/svn/ and svn_module to skolesys.}

\subsection{ISO and tar.gz based modules\label{skolesys-deb-iso}}
There is not much to say about these fetch methods. All skolesys-deb needs to know is where to find the iso or tgz files. Like when using subversion for fetch method you can set environment variables that will serve as default directories for iso/tgz file hunting.

command line options -l iso_dir and -t  tgz_dir can also be used to set the directory location and these will of course override the environment variables if set. 

\begin{tableii}{c|l}{}{Environment Variable}{Description}
  \lineii{\member{SKOLESYS_ISODIR}}{The default directory when searching for ISO files.}
  \lineii{\member{SKOLESYS_TGZDIR}}{The default directory when searching for tgz files.}
\end{tableii}

\section{skolesys-apt\label{skolesys-apt}}
After creating debian packages the next thing needed is to make life easier for debian users who want to use the packages. To do that we are going to create a standard debian package repository which debian APT (Advanced Package Tool) relies on. 

skolesys-apt can do this automatically, all we need to do is create a simple aptinfo file and place the debian files in a distribution hierarchi - also we need a gpg key-pair to sign the APT Release file. If you don't have such a key-pair it will be shown how to gererate it.

skolesys-apt will create the APT repository on the local filesystem ready to ship by copying the two folders \member{dists} and \member{pool} and the public part of the gpg key-pair to the host running the webserver that is going to expose the repository.

\index{skolesys-apt}
\member{Usage: skolesys-apt dist_name}

\subsection{Prepair gpg key-pair\label{skolesys-apt-gpg}}

Prerequisite: You need \member{gnupg} installed to go any further.

\subsubsection{Reuse an existing gpg key-pair\label{skolesys-apt-reusegpg}}
If you already have a gpg key-pair you wish to use for signing the APT repository it os possible to export the key-pair and then import it on the host where you are running skolesys-apt.
Let us say that the user gpg-owner on host gpg-source-host has the key-pair we want to export and the user skolesys-devel on host gpg-dest-host needs to import it:

Export key-pair:
\begin{enumerate}
 \item log on to the source host gpg-owner@gpg-source-host
 \item Export the gpg key-pair:
\begin{verbatim}
gpg-owner@gpg-source-host$ gpg --armor --export-secret-keys EEF2B7FA > secret.gpg.asc
gpg-owner@gpg-source-host$ gpg --armor --export EEF2B7FA > public.gpg.asc
\end{verbatim}
\end{enumerate}
\note{EEF2B7FA is the key ID. Use gpg --list-keys to view your key ID's}

Import key-pair:

\begin{enumerate}
 \item Copy the secret.gpg.asc and public.gpg.asc from the source host
 \item log on to the destination host skolesys-developer@gpg-dest-host
 \item Import the key-pair
\begin{verbatim}
skolesys-devel@gpg-dest-host$ gpg --allow-secret-key-import --import public.gpg.asc secret.gpg.asc
\end{verbatim}
\end{enumerate}

\member{public.gpg.asc} will be the public key block that you must expose to the public to make your signed APT repository validate and thereby accessible to remote debian based systems.

\subsubsection{Create a gpg key-pair\label{skolesys-apt-creategpg}}
If you don't have a gpg key-pair to use for this task then you need to create one now.
\begin{enumerate}
 \item Create the gpg key-pair
 \begin{verbatim}
gpg --gen-key

.... (I used the no-existing email: jakob@email.com)

pub   1024D/55117311 2007-06-21
	Key fingerprint = F461 E5AF 0DE8 B3E4 358D  DEB6 EC23 7CF4 5511 7311
uid                  Jakob Simon-Gaarde <jakob@email.com>
sub   2048g/FDD61F6E 2007-06-21
 \end{verbatim}
 \note{It is not required that you supply a passphrase you can leave it empty.}

 \item In the example above the newly generated key-pair has ID 55117311
 \item Extract a public key block	
 \begin{verbatim}
gpg --export --armor jakob@email.com > public.gpg.asc
 \end{verbatim}

 \member{public.gpg.asc} will be the public key block that you must expose to the public to make your signed APT repository validate and thereby accessible to remote debian based systems.

\end{enumerate}

\subsection{Creating a distribution hierarchi\label{skolesys-apt-pool}}
Most of the skolesys-deb modules rely on fetching their content from a subversion repository. It is possible to setup this repository once and for all by setting the SKOLESYS_SVNBASE environment variable. 

However, if you need to fetch content for different skolesys-deb modules from different repositories you can specify the repository for each module by setting the svn_repos variable in the debinfo file or you can do it by using the -r command line option. svn_repos will override SKOLESYS_SVNBASE, -r command line options overrides everything.
The debinfo variable svn_module specifies the path to fetch from inside the repository. If svn_module is unset skolesys-deb will try to fetch using the module_name as path.

\begin{tableii}{c|l}{}{Environment Variable}{Description}
  \lineii{\member{SKOLESYS_SVNBASE}}{The default subversion repository or repository parent dir.}
\end{tableii}

\note{If you need to fetch from the base of a repository you will have to set svn_repos to the parent dir of the repository - so if I need to fetch from the root of the repository located at svn.mydomain.org/srv/svn/skolesys I will need to set svn_repos to somthing like svn+ssh://svn.mydomain.org/srv/svn/ and svn_module to skolesys.}

\subsection{Creating aptinfo file\label{skolesys-apt-aptinfo}}
There is not much to say about these fetch methods. All skolesys-deb needs to know is where to find the iso or tgz files. Like when using subversion for fetch method you can set environment variables that will serve as default directories for iso/tgz file hunting.

command line options -l iso_dir and -t  tgz_dir can also be used to set the directory location and these will of course override the environment variables if set. 

\begin{tableii}{c|l}{}{Environment Variable}{Description}
  \lineii{\member{SKOLESYS_ISODIR}}{The default directory when searching for ISO files.}
  \lineii{\member{SKOLESYS_TGZDIR}}{The default directory when searching for tgz files.}
\end{tableii}



\documentclass{manual}

\title{SkoleSYS Developer Documentation}

\makeindex			% tell \index to actually write the .idx file


\begin{document}

\maketitle

\ifhtml
\chapter*{Front Matter\label{front}}
\fi

\begin{abstract}

\noindent
This document addresses the developer specific issues of SkoleSYS. It is
split up in two main parts - SkoleSYS library documentation and SkoleSYS
distribution tools (disttools).

SkoleSYS Library Documentation coveres the skolesys python package wich
is essentially all SkoleSYS logic e.g. client, server, configuration, SOAP
and more.

SkoleSYS Distribution Tools documents the three tools for deb-packaging,
APT distribution and install CD creation (skolesys-deb, skolesys-apt and 
skolesys-cd)

\end{abstract}

\tableofcontents

\chapter{SkoleSYS Distribution Tools (trunk/disttools)\label{disttools}}
The SkoleSYS Distribution Tools are just convenience tools I have made to make the job of distributing SkoleSYS fast and easy. That way I can keep the main focus on developing the SkoleSYS libraries and GUI. 

The \member{skolesys-apt} script can fetch content via subversion, tarballs or iso-files and place data where you want it redistributed on target filesystems. You can define symlinks and install-scripts (preinst, postinst, prerm, postrm) and setup file permissions. It is also possible to register a scripts to be run by \member{skolesys-apt} after content extraction (ie. pre-bytecompile python scripts). Once the debinfo files (control files for \member{skolesys-apt}) are in place building a file is as easy as executing one command - you can even setup automatic version ticking and group debian packages together by syncronizing their version number.

After building debian packages \member{skolesys-apt} is the tool for building an APT archive. You only need to setup one aptinfo file (control file for \member{skolesys-apt}) per distribution you target create a \member{distribution hieararchi}\footnote{A distribution hieararchi tells \member{skolesys-apt} which packages go into which distribution components and which computer architecture they are build for.}. \member{skolesys-apt} will automatically generate the \member{dists} and \member{pool} directory so they are ready to put on your distribution webserver.
\input{skolesys-deb}
\input{skolesys-apt}
%\input{skolesys-cd}


% \chapter{Debugging \label{debugging}}
%
% XXX Explain Py_DEBUG, Py_TRACE_REFS, Py_REF_DEBUG.

\chapter{SkoleSYS Package (trunk/system)\label{skolesys-pack}}
\input{skolesys-deb}
\input{skolesys-apt}


\input{devel.ind}			% Index -- must be last

\end{document}


			% Index -- must be last

\end{document}


			% Index -- must be last

\end{document}


			% Index -- must be last

\end{document}


