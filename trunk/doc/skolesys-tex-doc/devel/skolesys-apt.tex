\section{skolesys-apt\label{skolesys-apt}}
After creating debian packages the next thing needed is to make life easier for debian users who want to use the packages. To do that we are going to create a standard debian package archive which debian APT (Advanced Package Tool) relies on. 

skolesys-apt can do this automatically, all we need to do is create a simple aptinfo file and place the debian files in a distribution hierarchi - also we need a gpg key-pair to sign the APT Release file. If you don't have such a key-pair it will be shown how to gererate it. It will also be shown how to copy a gpg key/pair if you already have one on another server that you want to reuse.

skolesys-apt will create the APT archive on the local filesystem ready to ship by copying the two folders \member{dists} and \member{pool} and the public part of the gpg key-pair to the host running the webserver that is going to expose the archive.

\index{skolesys-apt}
\member{Usage: skolesys-apt [-g gpgid] <distribution_name>}

\begin{tableiii}{l|l|l}{command}{Short-option}{Long-option}{Description}
  \lineiii{-g \var{GPGID}}{\member{--gpgid=\var{GPGID}}}{Use a specific key for signing the APT archive}
\end{tableiii}

There are tree ways to setup the GPG ID of the key to use for signing. Command line option (\member{-r} or \member{---gpgid}), aptfile variable (\member{gpgid}) or by using the environment variable \member{SKOLESYS_GPGID} to set a default gpg id for signing. This also the precedens order. Thus environment varible will be overridden by the aptinfo variable which will be overridden by the command line option.

\begin{tableii}{c|l}{}{Environment Variable}{Description}
  \lineii{\member{SKOLESYS_GPGID}}{The default GPG ID for signing APT repositories}
\end{tableii}

\subsection{Prepair gpg key-pair\label{skolesys-apt-gpg}}

Prerequisite: You need \member{gnupg} installed to go any further.

\subsubsection{Reuse an existing gpg key-pair\label{skolesys-apt-reusegpg}}
If you already have a gpg key-pair you wish to use for signing the APT archive it is possible to export the key-pair and then import it on the host where you are running skolesys-apt.
Let us say that the user gpg-owner on host gpg-source-host has the key-pair we want to export and the user skolesys-devel on host gpg-dest-host needs to import it:

Export key-pair:
\begin{enumerate}
 \item log on to the source host gpg-owner@gpg-source-host
 \item Export the gpg key-pair:
\begin{verbatim}
gpg-owner@gpg-source-host$ gpg --armor --export-secret-keys EEF2B7FA > secret.gpg.asc
gpg-owner@gpg-source-host$ gpg --armor --export EEF2B7FA > public.gpg.asc
\end{verbatim}
\end{enumerate}
\note{EEF2B7FA is the key ID. Use \member{gpg ---list-keys} to view your key ID's}

Import key-pair:

\begin{enumerate}
 \item Copy the secret.gpg.asc and public.gpg.asc from the source host to the destination host
 \item log on to the destination host skolesys-developer@gpg-dest-host
 \item Import the key-pair
\begin{verbatim}
skolesys-devel@gpg-dest-host$ gpg --allow-secret-key-import --import public.gpg.asc secret.gpg.asc
\end{verbatim}
\end{enumerate}

\member{public.gpg.asc} will be the public key block that you must expose to the public to make your signed APT archive validate and thereby accessible to remote debian based systems.

\subsubsection{Create a gpg key-pair\label{skolesys-apt-creategpg}}
If you don't have a gpg key-pair to use for this task then you need to create one now.
\begin{enumerate}
 \item Create the gpg key-pair
 \begin{verbatim}
gpg --gen-key

.... (I used the non-existing email: jakob@email.com)

pub   1024D/55117311 2007-06-21
	Key fingerprint = F461 E5AF 0DE8 B3E4 358D  DEB6 EC23 7CF4 5511 7311
uid                  Jakob Simon-Gaarde <jakob@email.com>
sub   2048g/FDD61F6E 2007-06-21
 \end{verbatim}
 \note{It is not required that you supply a passphrase you can leave it empty.}

 \item In the example above the newly generated key-pair has ID 55117311
 \item Extract a public key block	
 \begin{verbatim}
gpg --export --armor jakob@email.com > public.gpg.asc
 \end{verbatim}

 \member{public.gpg.asc} will be the public key block that you must expose to the public to make your signed APT archive validate and thereby accessible to remote debian based systems.

\end{enumerate}

\subsection{Creating a distribution hierarchi\label{skolesys-apt-pool}}
In order to build the archive it is nessecary create a distribution hierarchi that can tell skolesys-apt how to group the packages. In a distribution you might need to divide your packages into components like \member{free}, \member{non-free}, \member{eyecandy} or \member{security}. And if you are targetting binary packages to more than one computer architecture you will need to place them in each there own platform branch (ie. \member{i386}, \member{amd64} or \member{all}). The hierarchi is build using the filesystem. Each aptinfo file target a distribution with their hierarchi created in a directory named after the distribution codename:

\citetitle{Pseudo distribution hierarchi:}
\begin{verbatim}
codename
codename/component-1
codename/component-1/architecture-1
codename/component-1/architecture-1/package1_arch1.deb
codename/component-1/architecture-1/package2_arch1.deb
codename/component-1/architecture-2/architecture-2
codename/component-1/architecture-2/package1_arch2.deb
codename/component-1/architecture-2/package2_arch2.deb
codename/component-2
codename/component-2/architecture-1
codename/component-2/architecture-1/package3_arch1.deb
codename/component-2/architecture-1/package4_arch1.deb
codename/component-2/architecture-2
codename/component-2/architecture-2/package3_arch2.deb
codename/component-2/architecture-2/package4_arch2.deb
\end{verbatim}

\citetitle{A realistic example of a distribution hierarchi:}
\begin{verbatim}
feisty
feisty/main
feisty/main/i386
feisty/main/i386/skolesys-ltsp_4.2-2-skolesys1_i386.deb
feisty/main/i386/skolesys-ltsp-localdev-kde_4.2-2-skolesys1_i386.deb
feisty/main/amd64
feisty/main/amd64/skolesys-ltsp_4.2-2-skolesys1_amd64.deb
feisty/main/amd64/skolesys-ltsp-localdev-kde_4.2-2-skolesys1_amd64.deb
feisty/main/all/python-skolesys-client_0.9.5-26-feisty_all.deb
feisty/main/all/python-skolesys-seeder_0.9.5-26-feisty_all.deb
feisty/main/all/skolesys-qt4_0.9.5-26-feisty_all.deb
feisty/eyecandy
feisty/eyecandy/i386
feisty/eyecandy/i386/beryl-ubuntu-i386.deb
feisty/eyecandy/amd64
feisty/eyecandy/amd64/beryl-ubuntu-amd64.deb
\end{verbatim}

\subsection{Creating aptinfo file\label{skolesys-apt-aptinfo}}
The last step of creating the APT archive is creating the aptinfo file for the distribution. The aptinfo file is the control file used by \member{skolesys-apt} to pickup the information needed to do the creation. Like skolesys-deb you have to use a certain naming convention <distribution_name>_aptinfo.py. Inside the aptinfo file is an important variable called \member{release_info} holding the archive details. 

\note{\member{codename} is used by \member{skolesys-apt} to target the distribution hierarchi of packages on the local filesystem (see previous section).}

\note{You can setup a GPG ID targetting a certain key to use for signing an archive pr. aptinfo file by setting the \member{gpgid} variable (see below) this will override the environment variable \member{SKOLESYS_GPGID} but not the command line option -g}

\begin{tableiii}{c|l|l}{textrm}{Name}{Type}{Description}
  \lineiii{\member{origin}}{\member{string}}{Your name or the name of your organization}
  \lineiii{\member{label}}{\member{string}}{A label describing the specific build of the archive}
  \lineiii{\member{suite}}{\member{string}}{The common name of the APT archive(ie. 'testing','stable')}
  \lineiii{\member{version}}{\member{string}}{Version number (ie. 0.9.4-3)}
  \lineiii{\member{description}}{\member{string}}{Description is used to describe the release. For instance 'testing' would contain a warning that the packages have problems.}
\end{tableiii}
\citetitle{The release_info variable (key/value description)}

\citetitle{Real example:}
\begin{verbatim}
$ cat feisty_aptinfo.py
release_info = {
        'origin': 'SkoleSYS',
        'label': 'SkoleSYS',
        'suite': 'stable',
        'version': '0.9.5-2',
        'codename': 'feisty',
        'description': '''
SkoleSYS for feisty 0.9.5-2. This archive contains the SkoleSYS libraries and
a binary package of LTSP 4.2 for the i386 architecture'''}

gpgid = "EEF2B7FA"
\end{verbatim}

\begin{tableii}{c|l}{}{Environment Variable}{Description}
  \lineii{\member{SKOLESYS_GPGID}}{The default GPG ID for signing APT repositories}
\end{tableii}
