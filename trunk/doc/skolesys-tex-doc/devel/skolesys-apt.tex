\section{skolesys-apt\label{skolesys-apt}}
After creating debian packages the next thing needed is to make life easier for debian users who want to use the packages. To do that we are going to create a standard debian package repository which debian APT (Advanced Package Tool) relies on. 

skolesys-apt can do this automatically, all we need to do is create a simple aptinfo file and place the debian files in a distribution hierarchi - also we need a gpg key-pair to sign the APT Release file. If you don't have such a key-pair it will be shown how to gererate it.

skolesys-apt will create the APT repository on the local filesystem ready to ship by copying the two folders \member{dists} and \member{pool} and the public part of the gpg key-pair to the host running the webserver that is going to expose the repository.

\index{skolesys-apt}
\member{Usage: skolesys-apt dist_name}

\subsection{Prepair gpg key-pair\label{skolesys-apt-gpg}}

Prerequisite: You need \member{gnupg} installed to go any further.

\subsubsection{Reuse an existing gpg key-pair\label{skolesys-apt-reusegpg}}
If you already have a gpg key-pair you wish to use for signing the APT repository it os possible to export the key-pair and then import it on the host where you are running skolesys-apt.
Let us say that the user gpg-owner on host gpg-source-host has the key-pair we want to export and the user skolesys-devel on host gpg-dest-host needs to import it:

Export key-pair:
\begin{enumerate}
 \item log on to the source host gpg-owner@gpg-source-host
 \item Export the gpg key-pair:
\begin{verbatim}
gpg-owner@gpg-source-host$ gpg --armor --export-secret-keys EEF2B7FA > secret.gpg.asc
gpg-owner@gpg-source-host$ gpg --armor --export EEF2B7FA > public.gpg.asc
\end{verbatim}
\end{enumerate}
\note{EEF2B7FA is the key ID. Use gpg --list-keys to view your key ID's}

Import key-pair:

\begin{enumerate}
 \item Copy the secret.gpg.asc and public.gpg.asc from the source host
 \item log on to the destination host skolesys-developer@gpg-dest-host
 \item Import the key-pair
\begin{verbatim}
skolesys-devel@gpg-dest-host$ gpg --allow-secret-key-import --import public.gpg.asc secret.gpg.asc
\end{verbatim}
\end{enumerate}

\member{public.gpg.asc} will be the public key block that you must expose to the public to make your signed APT repository validate and thereby accessible to remote debian based systems.

\subsubsection{Create a gpg key-pair\label{skolesys-apt-creategpg}}
If you don't have a gpg key-pair to use for this task then you need to create one now.
\begin{enumerate}
 \item Create the gpg key-pair
 \begin{verbatim}
gpg --gen-key

.... (I used the no-existing email: jakob@email.com)

pub   1024D/55117311 2007-06-21
	Key fingerprint = F461 E5AF 0DE8 B3E4 358D  DEB6 EC23 7CF4 5511 7311
uid                  Jakob Simon-Gaarde <jakob@email.com>
sub   2048g/FDD61F6E 2007-06-21
 \end{verbatim}
 \note{It is not required that you supply a passphrase you can leave it empty.}

 \item In the example above the newly generated key-pair has ID 55117311
 \item Extract a public key block	
 \begin{verbatim}
gpg --export --armor jakob@email.com > public.gpg.asc
 \end{verbatim}

 \member{public.gpg.asc} will be the public key block that you must expose to the public to make your signed APT repository validate and thereby accessible to remote debian based systems.

\end{enumerate}

\subsection{Creating a distribution hierarchi\label{skolesys-apt-pool}}
Most of the skolesys-deb modules rely on fetching their content from a subversion repository. It is possible to setup this repository once and for all by setting the SKOLESYS_SVNBASE environment variable. 

However, if you need to fetch content for different skolesys-deb modules from different repositories you can specify the repository for each module by setting the svn_repos variable in the debinfo file or you can do it by using the -r command line option. svn_repos will override SKOLESYS_SVNBASE, -r command line options overrides everything.
The debinfo variable svn_module specifies the path to fetch from inside the repository. If svn_module is unset skolesys-deb will try to fetch using the module_name as path.

\begin{tableii}{c|l}{}{Environment Variable}{Description}
  \lineii{\member{SKOLESYS_SVNBASE}}{The default subversion repository or repository parent dir.}
\end{tableii}

\note{If you need to fetch from the base of a repository you will have to set svn_repos to the parent dir of the repository - so if I need to fetch from the root of the repository located at svn.mydomain.org/srv/svn/skolesys I will need to set svn_repos to somthing like svn+ssh://svn.mydomain.org/srv/svn/ and svn_module to skolesys.}

\subsection{Creating aptinfo file\label{skolesys-apt-aptinfo}}
There is not much to say about these fetch methods. All skolesys-deb needs to know is where to find the iso or tgz files. Like when using subversion for fetch method you can set environment variables that will serve as default directories for iso/tgz file hunting.

command line options -l iso_dir and -t  tgz_dir can also be used to set the directory location and these will of course override the environment variables if set. 

\begin{tableii}{c|l}{}{Environment Variable}{Description}
  \lineii{\member{SKOLESYS_ISODIR}}{The default directory when searching for ISO files.}
  \lineii{\member{SKOLESYS_TGZDIR}}{The default directory when searching for tgz files.}
\end{tableii}
